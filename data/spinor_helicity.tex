\section{Spinor-helicity formalism}
The spinor-helicity formalism is widely used in amplitude literatures. 
One can refer to~\cite{Dixon:1996wi,Elvang:2013cua} for good introductions\footnote{The convention may change according to the signature}.
We present some useful properties under the convention used in QCD~\cite{Dixon:1996wi}.
\\\\
In a gauge theory, one might want to write the kinematic part of amplitudes in terms of the Lorentz invariants that can be formed from external momenta, \eg the square $p^2$ of a four-vector $p^\mu$.
However, in most interesting models, there are not only scalar fields but also particles which have spins. 
A more suitable representation of the Lorentz group is the spinor two-dimensional spinor representation.
We can trade a four-dimensional vector for a pair of spinors.
To be more explicit, let us consider the massless case.
Consider two spinors $\bar{u}(p)$ and $v(p)$ for an outgoing fermion and an outgoing anti-fermion of momentum $p$.
They satisfy the Dirac equation and we can decompose them into two parts of different helicity:
\begin{equation*}
v_+(p) = \begin{pmatrix}
|p]_\alpha \\ 0
\end{pmatrix} := \lambda_p^\alpha
\quad,\quad
v_-(p) = \begin{pmatrix}
0 \\ |p\rangle^{\dot{\alpha}}
\end{pmatrix} := \tilde{\lambda}_p^{\dot{\alpha}}
\quad,\quad
\bar{u}_-(p) = \begin{pmatrix} 
0, & \langle p |_{\dot{\alpha}})\end{pmatrix}
\quad,\quad
\bar{u}_+(p) = \begin{pmatrix} [ p|^\alpha, & 0 \end{pmatrix}
\end{equation*} 
where $v_\pm = \frac{1}{2}(1\pm\gamma_5)v(p)$ (analog for the others) and $\alpha = 1,2$.
The crossing symmetry is garanteed since $v_\pm = u_\mp$.
Using the Dirac representation for the gamma matrices, 
we have
\begin{equation*}
\slashed{p} = \begin{pmatrix}
0 & p_{a\dot{b}} \\ 
p^{\dot{a}b} & 0
\end{pmatrix}
\end{equation*}
where 
\begin{equation*}
p^{\dot{a}b} = p_\mu(\bar{\sigma^\mu})^{\dot{a}b}
\quad,\quad
p_{a\dot{b}} = p_\mu(\sigma^\mu)_{a\dot{b}}
\end{equation*}
With a slight abuse of notation, the massless Dirac equation implies
\begin{equation*}
\slashed{p}v_\pm (p) = \slashed{p}|p^\pm\rangle = 0
\quad \mathrm{where}\quad
|p^+\rangle := |p]
\quad
|p^-\rangle := |p\rangle
\end{equation*}
The raising and lowering of spinor indices are achieved by the Levi-Civita tensor
\begin{equation*}
[p|^a = \epsilon^{ab}|p] \quad,\quad
|p\rangle^{\dot{a}} = \epsilon^{\dot{a}\dot{b}}\langle p |_{\dot{b}}
\end{equation*}
The spinor products defined in the following are Lorentz invariant objects
\begin{equation*}
\begin{split}
& \langle ij \rangle = \epsilon^{ab}(\lambda_i)_a(\lambda_j)_b = \bar{u}_-(i)u_+(j)
\\
& [ij] = \epsilon^{\dot{a}\dot{b}}(\tilde{\lambda}_i)_{\dot{a}}(\tilde{\lambda}_j)_{\dot{b}} = \bar{u}_+(i)u_-(j)
\end{split}
\end{equation*}
In this formalism, the momentum can also be chosen complex. 
In the case where the momentum is real, we note that
$(\slashed{p})^* = {}^t\slashed{p}$, which correspond to an exchange of the right and the left spinors. 
In other words, the chirality flip of all spinors in a spinor product is simply the complex conjugate of teh spinor product
\begin{equation*}
[ij] = \langle ij \rangle^* \quad\textrm{when both momenta are real}
\end{equation*}
The spinor products are antisymmetric and the product of an angle and an square spinors vanishes. 
With this remark and a slight abuse of notation, one has, from the completeness relation of spinors,
\begin{equation*}
\slashed{p} = |p\rangle [p| + |p]\langle p|
\end{equation*}
For computational purpose, we give here two useful identities
\begin{enumerate}
\item (Fierz) $\langle 1 |\gamma^\mu |2]\langle 3 |\gamma^\mu|4] = 2\langle 13 \rangle [24]$
\item (Schouten) $\langle ij \rangle \langle lk \rangle + \langle ik\rangle \langle jl\rangle + \langle il \rangle \langle kj \rangle$
\end{enumerate}
For the first one, the properties of the Pauli matrices are used and the second one can be derived from the fact that the spinors used here are two-dimensional objects.
%
%
In presence of massless vectors, we can write the corresponding polarization vectors by choosing an arbitrary reference vector $q$ which is not collinear to the momentum $p_i$ of the vector field:
\begin{equation*}
\begin{spilt}
& \varepsilon^+_\mu (p_i, q) = \frac{\langle q | \gamma |i]}{\sqrt{2}\langle qi \rangle}
\\
& \varepsilon^-_\mu (p_i, q) = -\frac{[q|\gamma_\mu | i\rangle}{\sqrt{2}[qi]}
\end{split}
\end{equation*}
One can check that these polarization vectors are transverse and give us the expected results from Feynman diagram computations. 
It might still be complicated if one has to recall the polarization vectors each time.
However, amplitudes can be computed without using the explicit expression of the polarization vectors.
%
%
\subsection*{3-point amplitudes}
The first non-trivial object to compute at tree-level is the three-point function $A_3[1^{h_1}2^{h_2}3^{h_3}]$, where the $h_i$'s denote the helicities.
By momentum conservation, we have $p_3^2 = (p_1 + p_2)^2 = \langle 12\rangle[21] = 0$. 
If we allow complex momenta, the spinor products can be non simultaneously vanishing (since they are not related by complex conjugation anymore). 
As a result, either the square or the angle product vanishes.








