\section{Spinor-helicity formalism}
The spinor-helicity formalism is widely used in amplitude literatures. 
One can refer to~\cite{Dixon:1996wi,Elvang:2013cua} for good introductions\footnote{The convention may change according to the signature}.
We present some useful properties under the convention used in QCD~\cite{Dixon:1996wi}.
\\\\
In a gauge theory, one might want to write the kinematic part of amplitudes in terms of the Lorentz invariants that can be formed from external momenta, \eg the square $p^2$ of a four-vector $p^\mu$.
However, in most interesting models, there are not only scalar fields but also particles which have spins. 
A more suitable representation of the Lorentz group is the spinor two-dimensional spinor representation.
We can trade a four-dimensional vector for a pair of spinors.
To be more explicit, let us consider the massless case.
Consider two spinors $\bar{u}(p)$ and $v(p)$ for an outgoing fermion and an outgoing anti-fermion of momentum $p$.
They satisfy the Dirac equation and we can decompose them into two parts of different helicity:
\begin{equation*}
v_+(p) = \begin{pmatrix}
|p]_\alpha \\ 0
\end{pmatrix} := \lambda_p^\alpha
\quad,\quad
v_-(p) = \begin{pmatrix}
0 \\ |p\rangle^{\dot{\alpha}}
\end{pmatrix} := \tilde{\lambda}_p^{\dot{\alpha}}
\quad,\quad
\bar{u}_-(p) = \begin{pmatrix} 
0, & \langle p |_{\dot{\alpha}})\end{pmatrix}
\quad,\quad
\bar{u}_+(p) = \begin{pmatrix} [ p|^\alpha, & 0 \end{pmatrix}
\end{equation*} 
where $v_\pm = \frac{1}{2}(1\pm\gamma_5)v(p)$ (analog for the others) and $\alpha = 1,2$.
The crossing symmetry is garanteed since $v_\pm = u_\mp$.
Using the Dirac representation for the gamma matrices, 
we have
\begin{equation*}
\slashed{p} = \begin{pmatrix}
0 & p_{a\dot{b}} \\ 
p^{\dot{a}b} & 0
\end{pmatrix}
\end{equation*}
where 
\begin{equation*}
p^{\dot{a}b} = p_\mu(\bar{\sigma^\mu})^{\dot{a}b}
\quad,\quad
p_{a\dot{b}} = p_\mu(\sigma^\mu)_{a\dot{b}}
\end{equation*}
With a slight abuse of notation, the massless Dirac equation implies
\begin{equation*}
\slashed{p}v_\pm (p) = \slashed{p}|p^\pm\rangle = 0
\quad \mathrm{where}\quad
|p^+\rangle := |p]
\quad
|p^-\rangle := |p\rangle
\end{equation*}
The raising and lowering of spinor indices are achieved by the Levi-Civita tensor
\begin{equation*}
[p|^a = \epsilon^{ab}|p] \quad,\quad
|p\rangle^{\dot{a}} = \epsilon^{\dot{a}\dot{b}}\langle p |_{\dot{b}}
\end{equation*}
The spinor products defined in the following are Lorentz invariant objects
\begin{equation*}
\begin{split}
& \langle ij \rangle = \epsilon^{ab}(\lambda_i)_a(\lambda_j)_b = \bar{u}_-(i)u_+(j)
\\
& [ij] = \epsilon^{\dot{a}\dot{b}}(\tilde{\lambda}_i)_{\dot{a}}(\tilde{\lambda}_j)_{\dot{b}} = \bar{u}_+(i)u_-(j)
\end{split}
\end{equation*}
In this formalism, the momentum can also be chosen complex. 
In the case where the momentum is real, we note that
$(\slashed{p})^* = {}^t\slashed{p}$, which correspond to an exchange of the right and the left spinors. 
In other words, the chirality flip of all spinors in a spinor product is simply the complex conjugate of teh spinor product
\begin{equation*}
[ij] = \langle ij \rangle^* \quad\textrm{when both momenta are real}
\end{equation*}
The spinor products are antisymmetric and the product of an angle and an square spinors vanishes. 
With this remark and a slight abuse of notation, one has, from the completeness relation of spinors,
\begin{equation}\label{p_splinors}
\slashed{p} = |p\rangle [p| + |p]\langle p|
\end{equation}
For computational purpose, we give here two useful identities
\begin{enumerate}
\item (Fierz) $\langle 1 |\gamma^\mu |2]\langle 3 |\gamma^\mu|4] = 2\langle 13 \rangle [24]$
\item (Schouten) $\langle ij \rangle \langle lk \rangle + \langle ik\rangle \langle jl\rangle + \langle il \rangle \langle kj \rangle$
\end{enumerate}
For the first one, the properties of the Pauli matrices are used and the second one can be derived from the fact that the spinors used here are two-dimensional objects.
%
%
In presence of massless vectors, we can write the corresponding polarization vectors by choosing an arbitrary reference vector $q$ which is not collinear to the momentum $p_i$ of the vector field:
\begin{equation*}
\begin{split}
& \varepsilon^+_\mu (p_i, q) = \frac{\langle q | \gamma |i]}{\sqrt{2}\langle qi \rangle}
\\
& \varepsilon^-_\mu (p_i, q) = -\frac{[q|\gamma_\mu | i\rangle}{\sqrt{2}[qi]}
\end{split}
\end{equation*}
One can check that these polarization vectors are transverse and give us the expected results from Feynman diagram computations. 
%
%
\subsection*{3-point kinematics}
The first non-trivial objects to compute at tree-level are three-point amplitudes $A_3[1^{h_1}2^{h_2}3^{h_3}]$, where the $h_i$'s denote the helicities.
We can compute them directly by applying Feynman rules. 
We list here the only two non-vanishing three-point gluon amplitudes (the coupling constant is omitted)
\begin{equation*}
\begin{split}
& A^{\mathrm{tree}}_3[g_1^- g_2^- g_3^+] = \frac{\langle 12 \rangle^3}{\langle 13 \rangle \langle 32 \rangle}
\\
& A^{\mathrm{tree}}_3[g_1^+ g_2^+ g_3^-] = \frac{ [12]^3}{[13 ][ 32 ]}
\end{split}
\end{equation*}
Meanwhile, their forms may be guessed even before explicit calculations.  
By momentum conservation, we have $p_3^2 = (p_1 + p_2)^2 = \langle 12\rangle[21] = 0$. 
If we allow complex momenta, the spinor products can be non simultaneously vanishing (since they are not related by complex conjugation anymore). 
As a result, either the square or the angle product vanishes.
Thus, a 3-point amplitude contains only square products or angle products. 
Let us take the example of $A_3[1^{-}2^{-}3^{-}]$
We may write an ansatz for it :
\begin{equation*}
A_3[1^{-}2^{-}3^{-}] = c\langle 12 \rangle^{x_{12}}\langle 23 \rangle^{x_{23}}\langle 31 \rangle^{x_{31}}
\end{equation*}
\cref{p_splinors} implies that the momentum $p$ remains invariant under the following transformations, called \textbf{little group scaling}
\begin{equation*}
|p\rangle \rightarrow t| p\rangle \quad,\quad 
|p] t^{-1}\rightarrow |p]
\end{equation*} 
A consequence of this is the scaling rule for amplitudes while an external leg is rescaled
\begin{equation*}
A_n = (\{ |1\rangle, |1], h_1\},\ldots,\{t_i|i\rangle, t_i^{-1}|i], h_i\},\ldots) = 
t_i^{-2h_i}A_n(\ldots, \{|i\rangle,|i], h_i\},\ldots)
\end{equation*}
Therefore, in the case of all-gluon, we have
\begin{equation*}
A_3[g_1^-g_2^-g_3^+] = c\frac{\langle 12\rangle^3}{\langle 23 \rangle\langle 31\rangle}
\end{equation*}
This corresponds to the answer obtained from direct computations.
%
%
\subsection*{Maximally-helicity-violating amplitudes and the Taylor-Parke formula for gluons}
Consider now $n$-point all-gluon amplitudes for $n\geq 4$. 
By playing around with the polarization vectors and the reference vectors in their spinor expression, we can show (see p.31 of~\cite{Elvang:2013cua}) that the following all-gluon amplitudes vanishes at tree-level
\begin{equation*}
A_n[g_1^\pm \ldots g_n^\pm] = A^{\mathrm{tree}}_n[g_1^\pm \ldots g_i^\mp \ldots g_n^\pm] = 0
\end{equation*}
for $n\geq 4$.
In other words, when all the external legs have the same helicity or only one is different from the others, the tree-level amplitude vanishes.
The first class of non-vanishing tree amplitudes is said to be \textbf{maximally-helicity-violating (MHV)}, which have at least two external legs have helicity different from the rest.
The MHV gluon amplitudes can be written in a surprisingly simple form
\begin{equation*}
\begin{split}
& A_n[1^+\ldots i^-\ldots j^-\ldots n^+] = 
\frac{\langle ij \rangle^4]}{\langle 12 \rangle\ldots \langle n1 \rangle}
\\
& A_n[1^-\ldots i^+\ldots j^+\ldots n^-] = 
\frac{[ ij]^4}{[ 12 ]\ldots [n1 ]}
\end{split}
\end{equation*}
this is called the \textbf{Parke-Taylor formula}, first conjectured in~\cite{PhysRevLett.56.2459} and later proven by the Berends-Giele off-shell recursion~\cite{BERENDS1988759}. 
A more recent approach to proving this formula is the BCFW on-shell recursion, which will be presented in the section. 






