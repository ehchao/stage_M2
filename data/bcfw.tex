\section{BCFW on-shell recursion}\label{sect-bcfw}
Tree-level amplitudes can be computed in using the singularities which appear when a propagator is put on-shell. 
Due to Britto, Cachazo, Feng and Witten~\cite{BRITTO2005499, PhysRevLett.94.181602}, one can establish a recursion relation for tree-level amplitudes by complex analysis (Cauchy's theorem).
The main idea is to take the external momenta to be complex and apply well chosen shifts to them.
This method does not require the knowledge of individual contribution of each diagram.
Let us see how this works in following~\cite{Elvang:2013cua}.
\\\\
Consider a tree-level n-point amplitude $A_n$ with massless external momenta.
The momenta of the external particles are $p_i$ for $i=1, \ldots, n$ with $p^2_i = 0$. 
The momentum conservation implies $\sum_{i=1}^n p_i^\mu = 0$. 
\\
Now, introduce $n$ complex vectors $r^{\mu}_i$ such that
\begin{enumerate}
\item $\sum_{i=1}^n r_i^{\mu}  = 0$
\item $r_i\cdot r_j = 0 \quad,\quad\forall i,j$
\item $p_i \cdot r_i = 0 \quad,\quad \forall i$
\end{enumerate}
We define the shifted momenta
\begin{equation*}
\hat{p}^\mu_i = p_i^{\mu} + z r_i^{\mu} \quad\textrm{with}\quad z\in\mathbb{C}
\end{equation*} 
It is easy to check that the momentum conservation holds for the shifted momenta and that the shifted momenta are themselves also null vectors.
We denote by $\hat{A}_n(z)$ the $n$-point amplitude computed in replacing the external momenta $p_i$'s in $A_n$ by the shifted ones. 
Note that $\hat{A}_n(z)$ is holomorphic and contains only simple poles $z_I$ in $z$\footnote{As one can see from a simple drawing, double or higher poles can not exist since they come from the propagators contained in an amplitude.}.
Then, by Cauchy's theorem, the unshifted amplitude $A_n = \hat{A}_n(z=0)$ is related to the shifted one by
\begin{equation}\label{residue}
A_n = - \sum_{z = z_I}\res\frac{\hat{A}_n(z)}{z} + B_n
\end{equation} 
where $B_n$ is the residue of the pole at $z = \infty$.
\\
There are some particular choices of shift such that $B_n = 0$.
Under these shifts, it suffices to compute only the first term of the above equation. 
Singularities appear when propagators are put on-shell. 
In the terms of the shifted momenta, they correspond to the poles of $\hat{A}(z)$.
Therefore, at tree-level, the poles of $\hat{A}(z)$ are the points where, for a non-trivial subset of generic momenta $\{p_i\}_{i\in I}$, 
\begin{equation*}
\hat{P}_I^2 := \big( \sum_{i\in I} \hat{p}_i \big)^2 = 0
\end{equation*}   
%
Then, one can check that
\begin{equation}\label{cauchy}
\res_{z=z_I}\frac{\hat{A}_n(z)}{z} = - \hat{A}_L(z_I)\frac{1}{P_I^2}\hat{A}_R(z_I)
\end{equation}
where 
\begin{equation*}
P_I := \sum_{i\in I}p_i
\end{equation*}
and $\hat{A}_{L,R}$ are two on-shell amplitudes with less external momenta connected by the on-shell propagator at two different ends.
\\
The BCFW recursion consists in applying a $[i,j\rangle$-shift :
\begin{equation}
|\hat{i}] = |i] + z |j], \quad |\hat{j}] = |j], \quad|\hat{i}\rangle = |i\rangle, \quad |\hat{j}\rangle = |j\rangle - z|i\rangle
\end{equation}  
with the rest of spinors remaining the same, 
and using the holomorphic property that we just reviewed.
\\
The term $B_n$ in~\cref{residue} will disappear when the shift is apply to ajacent lines $i,j$ for the following helicities~\cite{ArkaniHamed:2008yf}
\begin{equation*}
[-,-\rangle \quad [-,+\rangle \quad [+,+\rangle
\end{equation*}
Therefore, the unshifted amplitude can be computed by dropping the residue at infinity in~\cref{residue} and its first term on the right-hand side is given by~\cref{cauchy} under the BCFW shift.
%
\subsection{Example: $A_6[1^-2^-3^-4^+5^+6^+]$}





