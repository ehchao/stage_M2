\section{Forumlae}

\paragraph{Beta function}
\begin{equation}
B(a,b) = \int_0^1 \dd x (1-x)^{a-1}x^{b-1} = \frac{\Gamma(a)\Gamma(b)}{\Gamma(a+b)}
\end{equation}

\paragraph{Hypergeometric function}
\begin{equation}
_2F_1(a,b,c;z) = \frac{\Gamma(c)}{\Gamma(b)\Gamma(c-b)}\int_0^1\frac{t^{b-1} (1-t)^{c-b-1}}{(1-tz)^a} \dd t
\end{equation}
%
\begin{equation}
\begin{split}
_2F_1(a,b,c;z) & = (1-z)^{-a}(_2F_1)(a,c-b,c;\frac{z}{z-1})
\\
& = (1-z)^{-b}(_2F_1)(c-a,b,c;\frac{z}{z-1})
\\
& = (1-z)^{c-a-b}(_2F_1)(c-a,c-b,c;z)
\end{split}
\end{equation}
%
\begin{equation}
_2F_1(a,b,c;z) = 1 + \frac{ab}{1!c} + \frac{a(a+1)b(b+1)}{2!c(c+1)}z^2 + \ldots
=\sum_{n=0}^{\infty}\frac{(a)_n(b)_n}{(c)_n}\frac{z^n}{n!}
\end{equation}


\paragraph{$D$-dimensional sphere ($(D-1)$-sphere) measure }
\begin{equation}
\int \dd \Omega_{D} = \int^\pi_0 \dd \theta_{D-1}\sin^{D-2} \theta_{D-1} \times \ldots \times \int^\pi_0 \dd \theta_2 \sin\theta_2 \int^{2\pi}_0 \dd \theta_1
\end{equation}

\begin{equation}
\Omega_D = \int \dd \Omega_D = \frac{2\pi^{D/2}}{\Gamma(\frac{D}{2})}
\end{equation}
%wrong....different notations
%\begin{equation}
%\int\dd \Omega_{D+1} = 2\pi^{D-2}\frac{1}{\Gamma(\frac{1}{2}D)}\int^{\pi}_0\dd \theta_D \sin^{D-1}\theta_D
%\end{equation}

%\begin{equation}
%\Omega_{D+1} = 2^D \pi^{D/2} \frac{\Gamma(\frac{1}{2}D)}{\Gamma(D)}
%\end{equation}

\paragraph{Gamma function}
\begin{equation}
\Gamma(\epsilon) = \frac{1}{\epsilon} - \gamma_E + \mathcal{O}(\epsilon)
\end{equation}

\paragraph{Feynman parametrization}
\begin{equation}
\frac{1}{\prod_{i=1}^n A_i} = \int_0^1 \dd x_1 \ldots\int^1_0\dd x_n\frac{(n-1)!\delta(\sum_{i=1}^n x_i -1)}{(\sum_{i=1}^n x_i A_i)^n}
\end{equation}

\paragraph{Useful integrals}
\begin{equation}
\int\frac{\dd^D k}{(2\pi)^D}\frac{k^{2a}}{(k^2-\Delta)^b} = i(-1)^{a-b}\frac{1}{(4\pi)^{D/2}\Delta^{b-a-D/2}}\frac{\Gamma(a+D/2)\Gamma(b-a-D/2)}{\Gamma(b)\Gamma(D/2)}
\end{equation}

\paragraph{3-point MHV amplitudes}
\begin{equation}
\begin{split}
& A_3[1^-2^-3^+] = \frac{\langle 1 2 \rangle^3}{\langle 23 \rangle \langle 31 \rangle}
\\
& A_3[1^+2^+3^-] = \frac{[12]^3}{[23][31]}
\end{split}
\end{equation}
\paragraph{Parke-Taylor}
\begin{equation}
A_n[1^+\ldots i^-\ldots j^-\ldots n^+] = \frac{\langle i j \rangle^4}{\langle 12\rangle \ldots\langle n 1 \rangle}
\end{equation}