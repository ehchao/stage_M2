\section{Basis for master integrals}
It is known that there exists a basis for one-loop integrals: one can write down a one-loop integral in terms of linear combination of box, triangle, bubble and tadpole integrals. 
Let us follow the review done in~\cite{Gluza:2010ws}.
\\\\
We consider integrals in $D= 4-2\epsilon$ dimension and keep external momenta to be four-dimensional.
A generic loop-integral can be written as
\begin{equation*}
I_n[P(l)] = 
-i\int\frac{\dd^D l}{(2\pi)^D}\frac{P(l)}{l^2(l-K_1)^2(l-K_{12})^2\ldots(l-K_{1\ldots (n-1)})^2}
\end{equation*}
where $P$ is a polynomial in $l$.
Let us begin with integrals with high multiplicity, with five or more external legs.
In this case, we can use the Gram determinant defined as
\begin{equation*}
\begin{split}
& G\begin{pmatrix}
p_1,\ldots, p_l \\
q_1,\ldots, q_l 
\end{pmatrix}
:= \det(2p_i\cdot q_j)
\\
& G(p_1, \ldots , p_l) := G\begin{pmatrix}
p_1,\ldots, p_l \\
p_1,\ldots, p_l 
\end{pmatrix}
\end{split}
\end{equation*} 
to reduce the integral.
In fact, the Gram determinant vanishes if either $\{p_i\}$ or $\{q_i\}$ are linearly dependent.
One can expand each of the four-vectors $v^\mu$ in a basis of four chosen external momenta $\{b_i\}_{i=1\ldots 4}$ with the help of Gram determinants.
At one-loop, the terms $l\cdot b_i$ are reducible since we can write them as differences between terms appearing in the denominator.
They will lead to sums of integrals with fewer propagators or fewer powers in $l$. 
In other words, 
\begin{equation*}
I_n[(l\cdot v)^n] \rightarrow I_{n-1}[(l\cdot v)^{n-1}]\oplus I_n[(l\cdot v)^{n-1}]
\end{equation*}
Next, we have to know reduce the five- or higher-point integrals with trivial numerators. 
Let us first consider the reduction of six- or higher-point integrals, which can be done to all orders in $\epsilon$. 
\\\\
Since the external momenta are four-dimensional, 
\begin{equation*}
G\begin{pmatrix}
l & 1 & 2 & 3 & 4\\
5 & 1 & 2 & 3 & 4 
\end{pmatrix}
 = 0
\end{equation*}
Accordingly, we can show, by expanding the Gram determinant, that
\begin{equation*}
I_n\Big[G\begin{pmatrix}
l & 1 & 2 & 3 & 4\\
5 & 1 & 2 & 3 & 4 
\end{pmatrix}\Big]
 = 0
 \quad\quad n\leq 6
\end{equation*}
As for the five-point integral, there might be a worry about the $\mathcal{O}(\epsilon^0)$ term or even the divergent terms.
However, divergences appear only when the loop momentum $l$ is soft or collinear to one of the external momenta, where the five-point Gram determinant also vanishes, we have
\begin{equation*}
I_5[G(l,1,2,3,4)] = \mathcal{O}(\epsilon)
\end{equation*}
Hence, the five-point integral can be also reduced by expanding the Gram determinant.
The reduction of five- or higher-point integrals are given explicitly in~\cite{Gluza:2010ws}









