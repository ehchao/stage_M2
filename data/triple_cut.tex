\section{Triangle and bubble coefficients in $D=4$}
Forde proposed a method using triple cut to determine the triangle and bubble coefficients in one-loop in four-dimension~\cite{Forde:2007mi}.
%
\subsection*{Triangle coefficient} 
Upon applying a triple cut, the three delta functions leave only one degree of freedom in four-dimension. 
Let us denote it by $t$.
The most general cut expression takes the form 
\begin{equation}\label{tricut}
(2\pi)^3\int\dd l^4 \prod_{i=0}^2 \delta(l_i^2) A_1A_2A_3 = 
(2\pi)^3\int\dd l^4 \prod_{i=0}^2 \delta(l_i^2)\Big(\mathrm{Inf}_t(A_1A_2A_3)(t) + \sum_{\textrm{$t_j$ poles}}\frac{\res_{t = t_j}A_1 A_2 A_3}{t-t_j}\Big)
\end{equation}
where the $A$'s are tree amplitudes and the first term on the right-hand side is defined by
\begin{equation*}
\lim_{t\rightarrow\infty}\big([\mathrm{Inf}_tA_1A_2A_3](t) - A_1A_2A_3(t)\big)
\end{equation*}
In general, $\mathrm{Inf}_tA_1A_2A_3$ is some polynomial in $t$. 
In most of cases, the propagators of the type $\frac{1}{(l-P)^2}$ contain two poles that can be expressed in terms of $t$. 
The second part of~\cref{tricut} can hence be considered as contributions of cut boxes. 
As a result, the information on the bubble and triangle contributions is only contained in the first term on the \rhs of~\cref{tricut}.
According to Forde, we can choose a parametrization such that only the constant term in the polynomial $\mathrm{Inf}_t(A_1A_2A_3)(t)$ contributes to the integral. 
We will see how this works in the next paragraph.
%
\\\\
Consider a triple cut leading to the constraints
\begin{equation*}
l^2 = (l-K_1)^2 = (l-K_2)^2 = 0
\end{equation*}
One can construct two null vectors $K_1^\flat$ and $K_2^\flat$ from the external momenta $K_1$ and $K_2$ 
\begin{equation*}
K_1^\flat = K_1 - \frac{S_1}{\gamma}K_2^\flat \quad\quad
K_2^\flat = K_2 - \frac{S_2}{\gamma}K_1^\flat
\end{equation*}
%
where $S_i = K_i^2$ and $\gamma = \langle K_1^\flat|K_2^\flat|K_1^\flat] =\langle K_2^\flat|K_1^\flat|K_2^\flat]$.
There are two possibilities for $\gamma$
\begin{equation*}
\gamma_{\pm} = (K_1\cdot K_2) \pm\sqrt{\Delta}\quad
\mathrm{where}\quad
\Delta = (K_1\cdot K_2)^2 - K_1^2K_2^2
\end{equation*}
In terms of the spinor components, the constraints require
\begin{equation*}
\langle l | = t\langle K_1^\flat| + \alpha_{01}\langle K_2^\flat| \quad\quad
[ l | = \frac{\alpha_{02}}{t}[ K_1^\flat| + [ K_2^\flat|
\end{equation*}
where
\begin{equation*}
\alpha_{01} = \frac{S_1(\gamma S_2)}{\gamma^2 - S_1S_2}\quad \quad
\alpha_{02} = \frac{S_2(\gamma S_1)}{\gamma^2 - S_1S_2}
\end{equation*}
%
and, as a four-vector,
\begin{equation*}
l^\mu = \alpha_{02} K_1^{\flat,\mu} + \alpha_{01}K_2^{\flat,\mu} + \frac{t}{2}\langle K_1^\flat | \gamma^\mu |K_2^\flat] + \frac{\alpha_{01}\alpha_{02}}{2t}\langle K_2^\flat|\gamma^\mu |K_1^\flat]
\end{equation*}
%
Now, we can use an argument similar to the one in~\cite{Ossola:2006us} on the spurious term to find the vanishing integrals.
To illustrate this argument, we consider the case of a rank-1 3-point-like integral. 
By simple arguments on the rank and the dependence on external momenta, 
\begin{equation*}
\int\dd^D q \frac{q^\mu}{D_0(q)D_1(q+p_1)D_2(q+p_2)} = c_1p_1^\mu + c_2p_2^\mu
\end{equation*}
If we contract the above relation with a vector $v^\mu$ orthogonal to $p_1$ and $p_2$, we obtain a vanishing integral.
$q\cdot v$ is hence a spurious term.
The same technique can be applied to show that $(q\cdot v)^n$ is spurious for any $n>0$. 
Back to our case, we can use the fact that $\langle K_1^{\flat,\pm} | K_{1,2}|K_2^{\flat,\pm}\rangle = 0 $ and $\langle K_1^{\flat,\pm}|\gamma^\mu|K_{2}^{\flat,\pm}\rangle\langle K_1^{\flat,\pm}|\gamma_\mu|K_{2}^{\flat,\pm}\rangle = 0$
to show that all only the constant term in the polynomial $\mathrm{Inf}_t(A_1A_2A_3)(t)$ contributes to~\cref{tricut}
%
%
\subsection*{Two-particle cuts and scalar bubble coefficients}
Triple cuts can also be used to determine bubble coefficients. 
Let us first consider a two-particle cut.
Contrary to the previous case, we will have two free parameters $y$ and $t$.
\begin{equation}\label{2-part_cut}
\begin{split}
(2\pi)^2\int \dd^4 l \prod_{i=0}^1 \delta(l_i^2) A_1 A_2 = & 
(2\pi)^2\int \dd t\dd y J_{t,y}\Big(\big[\mathrm{Inf}_t [\mathrm{Inf}_y(A_1A_2)](y)\big](t) 
\\&
+
\mathrm{Inf}_t\big(\sum_{\textrm{poles}}\frac{\mathrm{Res}_{y = y_i}A_1A_2}{y-y_j})
+
\mathrm{Inf}_t\big(\sum_{\textrm{poles}}\frac{\mathrm{Res}_{t = t_l}\mathrm{Inf}_y A_1A_2}{t-t_l})
\\ &
+
\mathrm{Inf}_t\big(\sum_{\textrm{poles}}\frac{\mathrm{Inf}_t\big(\sum_{\textrm{poles}}\frac{\mathrm{Res}_{y = y_i}A_1A_2}{y-y_j}\big)}{t-t_l}\big)
\Big)
\end{split}
\end{equation}
We parametrize the loop momentum $l$ by
\begin{equation*}
l^\mu = yK_1^{\flat,\mu} + \frac{S_1}{\gamma}(1-y)\chi^\mu + \frac{t}{2}\langle K_1^\flat|\gamma^\mu|\chi] + \frac{S_1}{2\gamma}\frac{y}{t}(1-y)\langle \chi|\gamma^\mu|K_1^\flat]
\end{equation*}
where
\begin{equation*}
K_1^\flat = K_1 - \frac{S_1}{\gamma}\chi^\mu ,\quad
\gamma = \langle \chi | \slashed{K}_1^\flat|\chi]
\end{equation*}
and $\chi$ is an arbitrary null vector (non-collinear to $K_1$).
As in the triangle case, we can use $\langle K_1^\flat| \slashed{K}_1|\chi] = \langle K_1^\flat|\slashed{K}_1 | \gamma^\mu|\chi]\langle K_1^\flat|\gamma_\mu|\chi]= 0$ to show that
\begin{equation*}
\int\dd t\dd y J_{t,y}t^n = \int \dd t \dd y \big(\frac{y}{t}\big)^n(1-y)^n = 0
\quad\Rightarrow\quad
\int \dd t \dd y J_{t,y}t^n = 0
\quad\textrm{for}\quad n\neq 0
\end{equation*} 
From the Passarino-Veltman reduction, we are able to conclude that all terms of the type $t^0y^m$ are non-vanishing, which makes difference \wrt the triple cut case. 
Naively, one might think that the bubble coefficients come from the first term of~\cref{2-part_cut}.
However, the loop momentum parametrization that has been used leads to more non-vanishing terms than in the triple cut case. 
Furthermore, the second and the third terms in~\cref{2-part_cut} can be reduced to scalar bubble integrals plus triangle integrals. 
In order to relate the contribution to the bubble coefficient of the residue piece, we apply an additional constraint: $(l+K_2)^2=0$ to the two-particle cut.
This will then fix the free parameter $y$.
The integral will then take the form
\begin{equation*}
(2\pi)^3\int \dd t \dd y J_t'\big(\delta(y-y_+) + \delta(y-y_-)\big) \mathcal{M}(y,t) = -2(2\pi)^2 i \int \dd t \dd y J_{t,y}\mathrm{Res}_{y = y_{\pm}\frac{\mathcal{M}(y,t)}{(l+K_2)^2}}
\end{equation*}
where $\mathcal{M}(y,t)$ is a general cut integral and $J_t'$ a function of $t$.
Therefore, the residue contributions are, up to a factor of $-\frac{1}{2}$, given by the triple cut.
As discussed in the previous subsection, the $t^0$ contribution in the triple cut gives the triangle coefficients ant the other terms in power of $t$ give the scalar bubble coefficients.
\\\\
We will just list here the final result without giving computational details which involve calculations of non-vanishing integrals and reduction techniques.
The bubble coefficient is given by
\begin{equation*}
b = -i[\Inf_t[\Inf_y A_1 A_2](y)](t)\big|_{t\rightarrow 0 , y^m\rightarrow \frac{1}{m+1}}
-\frac{1}{2}\sum_{C_{\mathrm{tri}}}[\Inf_t A_1A_2A_3](t)\big|_{t^j\rightarrow T(j)}
\end{equation*}
where the subscripts indicate the substitutions to be done and $T(j)$ is given in eq.(5.26) of~\cite{Forde:2007mi}. 
The second term above is summed over all possible triangle configurations.
\\\\
As an example, let us cite briefly how the two-mass linear triangle
\begin{equation*}
(2\pi)^2\int\dd^4 l \prod_{i=0}^1\delta(l_i^2)\frac{\langle K_2|\slashed{l}|K_1]}{(l+K_2)^2}
\end{equation*}
is treated in~\cite{Forde:2007mi}:
\begin{enumerate}
\item Do a two-particle cut. We will then get two terms in the integrand: one in $y^0$ and a residual term. The term in $y^0$ gives us a contribution to the bubble coefficients; while the residual term means that a triple cut is needed to get the contributions to the bubble and triangle contributions.
%
\item Apply a triple cut. There will be a term in $t$ if we choose the parametrization introduced previously, which encodes information on the bubble coefficients.
%
\item This term then requires the knowledge of $\int \dd t J_t' t$, which can be re-expressed in terms of our parametrization. Then, apply the Passarino-Veltman reduction to get the relationsship between $\int \dd tJ_t' t$ and the cut bubble $B_0^{\mathrm{cut}(K_1^2)}$
\end{enumerate}
%
%
%
\subsection*{Example: triangle coefficient extraction for six-photon amplitude in QED}
Let us show explicitly how to extract the coefficient for scalar triangle using the example of the six-photon amplitude $A_6[1^-2^+3^-4^+5^-6^+]$ in QED.
This example is given in~\cite{Forde:2007mi}.
\\\\
First, we have to compute the tree-level four-point amplitude $A_4[1^{h_1}2^{h_2}3^{h_3}4^{h_4}]$. 
Contrary to non-abelian theories, the four-photon amplitude vanishes at tree-level in QED.
Thus, the relevant four-point amplitudes are following ones which containt two photons and two fermions
\begin{equation*}
A_4^{\mathrm{tree}}(\bar{f}_1^- f_2^+ \gamma_1^-\gamma_2^+)
\quad,\quad
A_4^{\mathrm{tree}}(\bar{f}_1^+ f_2^- \gamma_1^-\gamma_2^+)
\end{equation*}
Recall that the polarization vectors contracted with the $\gamma$-matrices are represented by\footnote{This can be shown using the expression given previously and the Fierz identity.}
\begin{equation*}
\begin{split}
& \slashed{\epsilon}_+(p,k) = -\frac{\sqrt{2}}{\langle kp\rangle}\big(|p]\langle k| + |k\rangle[p|\big)
\\
& \slashed{\epsilon}_-(p,k) = \frac{\sqrt{2}}{[kp]}\big(|p\rangle [k| + |k]\langle p|\big)
\end{split}
\end{equation*}
We denote the out-going fermion momenta by $q_i$ and the out-going photon momenta by $p_i$. 
A direct Feynman diagram computation gives\footnote{Recall that we take all the particles out-going in our convention.}
\begin{equation*}
\begin{split}
A_{4}^{\mathrm{tree}}(\bar{f}_1^- f_2^+ \gamma_1^-\gamma_2^+) 
= &
-\frac{ie^2}{(p_2+q_2)^2}\langle -q_1, -p_1\rangle \frac{\sqrt{2}}{[k, -p_1]}
[i|\big(|p_2] \langle p_2| + |q_2]\langle q_2|\big)
\frac{\sqrt{2}}{\langle k p_2\rangle}
\big(|p_2]\langle k| + |k\rangle[p_2|\big) |q_2]
\\
&
-\frac{ie^2}{(-q_1-p_2)^2}\langle -q_1, k'\rangle
\frac{\sqrt{2}}{\langle k', p_2\rangle}[p_2|
\big( -|p_2]\langle p_2| - |q_1]\langle q_1|\big)
\frac{\sqrt{2}}{[k', -p_1]}|k'\rangle [-p_1, q_2]
\end{split}
\end{equation*}
The first term corresponds to the $s$-channel and the second one $u$-channel.
By choosing the reference momentum $k'$ to be $q_1$, the second term above vanishes. 
We can choose $k = q_2$ and get 
\begin{equation}\label{4pt_qed1}
\begin{split}
A_{4}^{\mathrm{tree}}(\bar{f}_1^- f_2^+ \gamma_1^-\gamma_2^+) 
= &
\frac{2ie^2\langle q_1 p_1 \rangle[p_2 q_2]^2}{[q_2 p_1](p_1+q_1)^2}
\\ 
= &
\frac{2ie^2[42]^2}{[23][31]}
\end{split}
\end{equation}
where in the last equation, the particles are indexed in the order in which they appear in the argument of the amplitude.
In the same manner, there is an $s$-channel and an $u$-channel contribution for the other relevant four-point amplitude. 
The $s$-channel can be made vanishing by choosing properly the reference momentum. 
If we choose the reference momentum for the $u$-channel to be $q_2$, we find 
\begin{equation}\label{4pt_qed2}
\begin{split}
A_4^{\mathrm{tree}}(\bar{f}_1^+ f_2^- \gamma_1^- \gamma_2^+) = & \frac{2ie^2}{(q_2+p_2)^2}\frac{\langle p_1q_2\rangle^2}{\langle p_2q_1\rangle\langle q_2p_2\rangle}
\\
= &
\frac{2ie^2\langle 32\rangle^2}{\langle 41 \rangle\langle 24\rangle}
\end{split}
\end{equation}
Applying a triple cut on the channel $12:34:56$, we get the two contributing cut amplitudes
\begin{equation}\label{triple_cut_1}
A_{4}^{\mathrm{tree}}(f_1^- 1^-2^+\bar{f}_3^+)
A_4^{\mathrm{tree}}(\bar{f}_1^+f_2^-5^-6^+)
A_4^{\mathrm{tree}}(\bar{f}_2^+ f_3^-3^-4^+)
=
-8e^6\frac{\langle l_1 1\rangle^2 \langle 5l_2 \rangle^2\langle 3l_3\rangle^2}{\langle2 l_3\rangle\langle l_1 2\rangle\langle 6 l_1 \rangle\langle l_2 6\rangle\langle 4 l_2\rangle\langle l_3 4\rangle}
\end{equation}
\begin{equation}\label{triple_cut_2}
A_{4}^{\mathrm{tree}}(f_1^+ 1^-2^+\bar{f}_3^-)
A_4^{\mathrm{tree}}(\bar{f}_1^-f_2^+5^-6^+)
A_4^{\mathrm{tree}}(\bar{f}_2^- f_3^+3^-4^+)
=
-8e^6\frac{[2l_1]^2[6l_2]^2[4l_3]^2}{[l_1 1 ][1l_3][l_2 5][5l_1][l_2 3][3l_3]}
\end{equation}
Under the triple cut parametrization,~\cref{triple_cut_1} and~\cref{triple_cut_2} lead to the same contribution. 
In fact, the ratio of~\cref{4pt_qed1} and~\cref{4pt_qed2} gives
\begin{equation*}
\end{equation*} 












