\subsection*{$A_5^{\mathrm{1-loop}}[1^-2^-3^+4^+5^+]$ in $\mathcal{N}=4$ super Yang-Mills}\label{sect-a5mhv}
Let us see how to use quadruple cuts to extract box coefficients by computing the all-gluon MHV amplitude $A_5^{\mathrm{1-loop}}[1^-2^-3^+4^+5^+]$ in $\mathcal{N}=4$ super Yang-Mills. 
There are five possible configurations for the external momenta under a quadruple cut.
Each configuration corresponds to a one-mass box integral.
Among the four tree-level amplitudes that we obtain from a quadruple cut~\cref{box_coeff}, there is a four-point amplitude and three three-point ones.
As already noted in the original quadruple cut paper~\cite{Britto:2004nc}, massless legs in quadruple cut would lead to inconsistent results if we choose loop-momentun to be real\footnote{This was handled in~\cite{Britto:2004nc} by working in the signature $(--++)$.}. 
However, in the contour integral scheme with complex loop momentum that we mentioned earlier, this problem is not present.
\\\\
Next, let us consider all the possible helicity configuration on the cuts.
To start, we may take all the internal particles to be gluons. 
Then, we relate the all-gluon tree-level amplitudes to other amplitudes with different species by~\cref{super_wi}. 
By the discussion given in section~\ref{sect-spinor}, we have to have exactly two particles of the same helicity in order to get non-vanishing contribution from three- and four-point tree-level amplitudes~\cref{a3mhv} and~\cref{parke-taylor}.
Up to this step, we have already eliminated a bunch of helicity configurations
\\\\
Finally, the on-shell condition of the internal propagtors may in certain case impose further relationships to the external momenta than just the momentum conservation.
These cases should also be discarded because they violate our requirement of non-collinearity of the external momenta. 
%\color{red}perhaps describe more with a graph.\color{black}
At the end of the day, there are only five possible solutions to the all-gluon configuration (figure~\ref{fig-a5mhv}).
We give the detailed computation hereunder.
\begin{figure}
  \centering
  \includegraphics[width=0.8\linewidth]{A5mhv.eps}
  \caption{The only non-vanishing contributions for $A_5^{\mathrm{1-loop}}[1^-2^-3^+4^+5^+]$ in $\mathcal{N}=4$ super Yang-Mills. The momenta in the loop are enumerated in the same way as the top-left one.}
  \label{fig-a5mhv}
\end{figure}
\paragraph{Configuration $c_1$}
The special three-particle kinematics tells us that either the square spinor product or the angle one of any two on-shell spinor linked to a trivalent vertex vanishes.
By considering the trivalent vertices appearing in this configuration, the only possible solution for the internal momenta in terms of spinors should satisfy
\begin{equation}
|l_1\rangle = \alpha |5\rangle \quad,\quad
|l_2\rangle = \delta |3\rangle \quad,\quad
|l_3\rangle = \gamma |l_3\rangle\quad,\quad
|l_3] = \eta|4]\quad,\quad
|l_4] = \beta |4]
\end{equation}
Hence, the coefficient determined by~\cref{box_coeff} is
\begin{equation}\label{coef_c_1}
\begin{split}
c_1 = &
\frac{1}{2}\frac{\langle 12 \rangle^4}{\langle 1 2 \rangle \langle 2 l_2\rangle \langle l_2 l_1 \rangle\langle l_1 1 \rangle}
\frac{[3l_3]^3}{[l_3 l_2][l_2 3]}
\frac{\langle l_3 l_4 \rangle^{3}}{\langle l_4 4 \rangle \langle 4 l_3\rangle}
\frac{[l_4 5 ]^3}{[5l_1][l_1 l_4]}
\\
= & 
\frac{1}{2}\frac{\langle 12 \rangle^3}{\langle 2 l_2 \rangle\langle l_2 l_1 \rangle\langle l_1 1\rangle}
\frac{[34]^2\langle l_4 4\rangle}{[l_3 l_2][l_2 3]\langle 4 l_3\rangle}
\frac{[l_4 5]^2}{[5l_1 ][l_1 l_4]}
[3 l_3]\langle l_3 l_4 \rangle [l_4 5]
\\
= &
\frac{1}{2}
\frac{\langle 12 \rangle^3[34]^2[45]}{\langle 15 \rangle\langle2|\slashed{K}_{3}|4]}
\\
= &
\frac{1}{2}\frac{\langle 12 \rangle^4 s_{34}s_{45}}{\langle 12 \rangle\langle 23 \rangle \langle 34\rangle \langle 45\rangle \langle 51\rangle}
\\
= &
\frac{1}{2}s_{34}s_{45}A_5^{\textrm{MHV-tree}}[12345]
\end{split}
\end{equation}
We list here the results for the other contributions. 
The explicit computations are given in Appendix~\ref{appendix-a5mhv}\footnote{
In fact, the knowledge of $c_1$ suffices to obtain the full amplitude.
In a supersymmetric theory, we can derive any amplitude from a super-amplitude using the superspace formulation (see \eg ref~\cite{Elvang:2013cua} for an introduction). 
In an $\mathcal{N} = 4$ theory, the relationship between an $n$-point MHV amplitude and the relevant super-amplitude $\mathcal{A}_n$ at all level is uniquely determined by the knowledge at tree-level~\cite{Elvang:2010xn}.
The MHV super-amplitude takes the form 
\begin{equation}
\mathcal{A}_n^{\mathrm{MHV}} = \frac{\delta^{8}(\tilde{Q})}{\langle 12 \rangle \ldots \langle n 1\rangle}
\end{equation}
where $\tilde{Q}$ represents the set of supersymmetry generators.
To get an MHV amplitude with the gluons $m_1$ and $m_2$ having negative helicity, it suffices to apply the appropriate Grassmann differential operators to the superamplitude, which will produce the factor $\langle m_1 m_2\rangle$ that was missing compared \cref{parke-taylor}.
Equivalently, all the helicity configurations in a given cut are related and thus, the knowledge of $c_1$ suffices for determining the whole amplitude. 
}
%
%%%%%%%%%%%%%%%%%%%%%%%%%%%%%%%%%%%%%%%%%%%%%%%%%
\iffalse
\footnote{
In fact, the knowledge of $c_1$ suffices to obtain the full amplitude.
In a supersymmetric theory, we can derive any amplitude from a super-amplitude using the superspace formulation (see \eg ref~\cite{Elvang:2013cua} for an introduction). 
In an $\mathcal{N} = 4$ theory, the relationship between an $n$-point MHV amplitude and the relevant super-amplitude $\mathcal{A}_n$ at all level is uniquely determined by the knowledge at tree-level and is given by~\cite{Elvang:2010xn} 
\begin{equation}
\mathcal{A}_n^{\mathrm{MHV}} = \delta^{8}(\tilde{Q})\frac{1}{\langle n-1, n\rangle^4}A_n^{\mathrm{tree}}[+,+,\ldots,+,-,-]
\end{equation}
where $\tilde{Q}$ represents the set of supersymmetry generators.
As one can notice from this expression, the term on the right hand side should be cyclic. 
As a result, to determine the five-point MHV amplitude, it suffices to take~\cref{coef_c_1}, multiply it with the box integral $I_1$ corresponding to the configuration, pull out the factor $\langle 12 \rangle^4$ and symmetrize the rest.
The result matches exactly~\cref{final_result_a5mhv}.
}
\fi
%
%%%%%%%%%%%%%%%%%%%%%%%%%%%%%%%%%%%%%%%%%%%%%%%%%%%%%%%%
.
\begin{equation}\label{remaining_a5}
\begin{split}
& c_8 = \frac{1}{2} s_{51}s_{45} A_5^{\textrm{MHV-tree}}[12345]
\\
& c_{10} = \frac{1}{2}s_{51}s_{12}A_5^{\textrm{MHV-tree}}[12345]
\\
& c_{11} = \frac{1}{2}s_{12}s_{23}A_5^{\textrm{MHV-tree}}[12345]
\\
& c_{16}= \frac{1}{2}s_{23}s_{34}A_5^{\textrm{MHV-tree}}[12345]
\end{split}
\end{equation}
%
%
%%%%%%%%%%%%%%%%%%%%%%%%%%%%%%%%%%%%%
%put in appendix
\iffalse
\paragraph{Configuration $c_{8}$}
By~\cref{box_coeff}, the box coefficient corresponding to this configuration is
\begin{equation}
\begin{split}
c_8 = & \frac{1}{2}
\frac{[l_1 l_2]^3}{[l_1 1][1l_2]}
\frac{\langle 2 l_2 \rangle^4}{\langle 2l_2 \rangle\langle l_2 l_3\rangle\langle l_3 3 \rangle\langle 32 \rangle}
\frac{[l_4 4 ]^3}{[l_4 l_3][l_3 4]}
\frac{\langle l_4 l_1 \rangle^3}{\langle l_4 5 \rangle\langle 5 l_1\rangle}
\\
= &
\frac{\langle 12 \rangle^3[1l_1 ]\langle l_1 5\rangle^2 [54]^3}{\langle l_1 l_3\rangle\langle l_3 3 \rangle\langle 32 \rangle [l_3 4 ]^2\langle 45\rangle}
\end{split}
\end{equation}
By the three-particle kinematics and the fact that spinors are two-dimensional objects, set
\begin{equation}\label{coef_c8}
|l_1] = \alpha |5]
\quad,\quad
|l_1\rangle = a|4\rangle + b|5\rangle
\quad,\quad
|l_3\rangle = \beta |4\rangle
\quad,\quad
|l_3] = c|4] + d|5]
\end{equation}
From the on-shell condition $(l_1 - K_1)^2 = 0$, we get
\begin{equation}
a\langle 14\rangle = -b\langle 15 \rangle
\end{equation}
and from another on-shell condition $(l_1 + K_{45})^2 = 0$, we get
\begin{equation}
K_{45}^2 = -b\alpha [5|\slashed{K}_4|5\rangle 
\end{equation}
Hence
\begin{equation}
a\alpha = \frac{\langle 15\rangle}{\langle 14\rangle}\quad,\quad
b\alpha = -1
\end{equation}
In the same manner, we determine the other coeffients in~\cref{coef_c8} by two other on-shell conditions:
\begin{equation}
\begin{split}
& (l_3 - K_{45})^2 = 0 \quad\Rightarrow\quad
K_{45}^2 = c\beta [4|\slashed{K}_5|4\rangle
\\
& (l_3 + K_{23})^2 = 0 \quad\Rightarrow\quad
K_{23}^2 = -c\beta [4|\slashed{K_{23}}|4\rangle - d\beta [5|\slashed{K}_{23}|4\rangle
\end{split}
\end{equation}
Therefore
\begin{equation}
c\beta = 1 \quad,\quad 
d\beta = \frac{K_{234}^2}{[51]\langle 14 \rangle}
\end{equation}
Putting all these results back, we get
\begin{equation}
c_8 = \frac{1}{2} s_{51}s_{45} \frac{\langle 12 \rangle^3 [15] [54]}{\langle 43 \rangle\langle 32 \rangle}=\frac{1}{2} s_{51}s_{45} A_5^{\textrm{MHV-tree}}[12345]
\end{equation}
We will be using this machinery to determine the rest of the coefficients.
%
\paragraph{Configuration $c_{10}$}
\begin{equation}
\begin{split}
c_{10} = &
\frac{1}{2}\frac{\langle 1l_2\rangle^3}{\langle l_2 l_1 \rangle\langle l_1 1 \rangle}
\frac{[l_2 l_3]^3}{[l_2 2 ][2 l_3]}
\frac{[34]^4}{[34][4 l_4][l_4 l_3][l_3 3]}
\frac{[l_4 5]^3}{[l_4 l_1][l_1 5]}
\\
= &
\frac{1}{2}\frac{\langle 12 \rangle^3[2l_3]^3[34]^3[l_4 5 ]^2}{\langle 51 \rangle[15][2l_3]^2\langle l_3 1\rangle [4 l_4][l_4 l_3][l_3 3]}
\end{split}
\end{equation}
Set
\begin{equation}
|l_3\rangle = \alpha |2\rangle \quad,\quad |l_3] = a|2] + b|5]
\quad,\quad
|l_4\rangle = \beta |5\rangle \quad,\quad |l_4] = c|2]+d|5]
\end{equation}
Then
\begin{equation}
\begin{split}
& (l_3 + K_{12})^2 = 0 \quad\Rightarrow\quad K_{12}^2 = - \alpha a [2|\slashed{K}_{12}|2\rangle - \alpha b [5|\slashed{K}_{12}|2\rangle
\\
& (l_3 - K_{34})^2 = 0 \quad\Rightarrow\quad K_{34}^2 = \alpha a [2|\slashed{K}_{34}|2\rangle + \alpha b [5|\slashed{K}_{34}|2\rangle
\end{split}
\end{equation}
which leads to
\begin{equation}
\alpha a = \frac{[5|\slashed{K}_{34}|5\rangle}{[25]\langle 52\rangle}
\\
\alpha b = \frac{[21]\langle 15\rangle}{[52]\langle 25\rangle}
\end{equation}
On the other hand
\begin{equation}
\begin{split}
& (l_4 - K_{45})^2 = 0 \quad\Rightarrow\quad
K_{15}^2 = \beta c[2|\slashed{K}_1|5\rangle + \beta d [5|\slashed{K}_1|5\rangle
\\
& (l_4 + K_{34})^2 = 0 \quad\Rightarrow\quad
K_{34}^2 = \beta c [2|\slashed{K}_1|5\rangle - \beta d [5|\slashed{K}_{34}|5\rangle
\end{split}
\end{equation}
gives
\begin{equation}
\beta c = - \frac{[51]\langle 12 \rangle}{[52]\langle 25 \rangle}
\quad,\quad
\beta d = - \frac{[2| \slashed{K}_{34}|2\rangle}{[5|\slashed{K}_2|5\rangle}
\end{equation}
The coefficient $c_{10}$ is then given by
\begin{equation}
c_{10} = 
\frac{1}{2}\frac{[21][34]^3[51]\langle 12\rangle^4 [52]^2 \langle 52\rangle[25]}{\big(-[42][51]\langle 12\rangle - [2|\slashed{K}_{34}|2\rangle [45]\big)
\big( K_{51}^2 K_{21}^2 - (2K_2\cdot K_{34})(2K_5\cdot K_{34})\big)
\big( [5|\slashed{K}_{34}|5\rangle [23] - [21]\langle 15\rangle [35]\big)
}
\end{equation}
Using Schouten identity and the momentum conservation, we can rewrite the terms in the denominator in a more compact way
\begin{equation}
\begin{split}
& -[42][51]\langle 12\rangle - [2|\slashed{K}_{34}|2\rangle [45] = 
-\langle 32\rangle[52][34]
\\
& K_{51}^2 K_{21}^2 - (2K_2\cdot K_{34})(2K_5\cdot K_{34}) = 
-s_{52}s_{34}
\\
& [5|\slashed{K}_{34}|5\rangle [23] - [21]\langle 15\rangle [35]
= -\langle 45\rangle[52][34]
\end{split}
\end{equation}
As a result
\begin{equation}
c_{10} = \frac{1}{2}s_{51}s_{12}A_5^{\textrm{MHV-tree}}[12345]
\end{equation}
%
\paragraph{Configuration $c_{11}$}
\begin{equation}
\begin{split}
c_{11} = &
\frac{1}{2}\frac{[l_1 l_2]^3}{[l_1 1][1l_2]}
\frac{\langle l_2 2 \rangle^3}{\langle 2 l_3 \rangle\langle l_3 l_2 \rangle}
\frac{[3l_4]^3}{[l_4 l_3][l_3 3]}
\frac{[45]^4}{[45][5l_1][l_1l_4][l_4 4]}
\\
= &
-\frac{1}{2}
\frac{[l_1 1]\langle 12 \rangle^3[3l_4]^2[45]^3}{[23]\langle l_1 2 \rangle\langle 23 \rangle[5l_1][l_1l_4][l_4 4]}
\end{split}
\end{equation}
%
Set
\begin{equation}
|l_1\rangle = \alpha| 1\rangle \quad,\quad
|l_1] = a|1] + b|3] \quad,\quad
|l_4\rangle = \beta|3\rangle 
| l_4] = c|1] + d|3]
\end{equation}
Then
\begin{equation}
\begin{split}
& (l_1- K_{12})^2 = 0 \quad\Rightarrow\quad K_{12}^2 = \alpha a [1|\slashed{K}_2|1\rangle + \alpha b [3|\slashed{K}_2|1\rangle
\\
& (l_1 + K_{45})^2 = 0 \quad\Rightarrow\quad
K_{45}^2 = -\alpha a [1|\slashed{K}_{45}|1\rangle - \alpha b [3|\slashed{K}_{45}|1\rangle
\end{split}
\end{equation}
so
\begin{equation}
\alpha a = \frac{[3|\slashed{K}_{12}|3\rangle}{[13]\langle 31\rangle}
\quad,\quad
\alpha b = \frac{[21]\langle 23\rangle}{[13]\langle 31\rangle}
\end{equation}
On the other hand
\begin{equation}
\begin{split}
& (l_4 + K_{23})^2 = 0 \quad\Rightarrow\quad K_{23}^2 = -\beta c[1|\slashed{K}_2|3\rangle - \beta d [3|\slashed{K}_2|3\rangle
\\
& (l_4 - K_{45})^2 = 0 \quad\Rightarrow\quad K_{45}^2 = \beta c [1|\slashed{K}_{45}|3\rangle + \beta d [3|\slashed{K}_{45}|3\rangle
\end{split}
\end{equation}
Hence
\begin{equation}
\beta c = \frac{[32]\langle 21\rangle}{[31]\langle 13\rangle}
\quad,\quad
\beta d = -\frac{[1|\slashed{K}_{23}|1\rangle}{[31]\langle 13\rangle}
\end{equation}
and
\begin{equation}
c_{11} = \frac{1}{2}\frac{[32][21][31]^4\langle 13\rangle\langle 12 \rangle^4 [45]^3}{\big( [3|\slashed{K}_{12}|3\rangle [51] + [21]\langle 23\rangle [53]\big)
\big( [32]\langle 21\rangle[14] - [1|\slashed{K}_{23}|1\rangle[34]\big)
\big( (2K_{12}\cdot K_3)(2K_{23}\cdot K_1) - K_{23}^2K_{21}^2\big)
}
\end{equation}
Again, using Schouten identity and momentum conservation, we obtain the following equations which simplify the denominator
\begin{equation}
\begin{split}
& [3|\slashed{K}_{12}|3\rangle [51] + [21]\langle 23\rangle [53] = -[31][54]\langle 43 \rangle 
\\
& [32]\langle 21\rangle[14] - [1|\slashed{K}_{23}|1\rangle[34] = -[13][45]\langle 51 \rangle
\\
&
(2K_{12}\cdot K_3)(2K_{23}\cdot K_1) - K_{23}^2K_{21}^2 = s_{45}s_{31}
\end{split}
\end{equation}
At the end
\begin{equation}
c_{11} = \frac{1}{2}s_{12}s_{23}A_5^{\textrm{MHV-tree}}[12345]
\end{equation}
%
%
\paragraph{Configuration $c_{16}$}
\begin{equation}
\begin{split}
c_{16} = & \frac{1}{2}
\frac{\langle l_2 1 \rangle^4}{\langle l_2 1 \rangle\langle 15 \rangle\langle 5 l_1 \rangle\langle l_1 l_2\rangle}
\frac{[l_2 l_3]^3}{[l_3 2 ][2 l_2]}
\frac{\langle l_3 l_4\rangle^3}{\langle l_3 3\rangle\langle 3 l_4\rangle}
\frac{[l_4 4 ]^3}{[4l_1][l_1l_4]}
\\
= & 
-\frac{1}{2}\frac{\langle l_2 1\rangle^3[l_2 2 ]\langle 23 \rangle^3[34]^3}{\langle 15 \rangle[4l_4]\langle l_4 l_2\rangle\langle 3l_2\rangle \langle 3 l_4\rangle\langle 54\rangle[4l_4]}
\end{split}
\end{equation}
Set
\begin{equation}
|l_2\rangle = \alpha|2\rangle \quad,\quad
|l_2] = a|2] + b|4] \quad,\quad
|l_4\rangle = \beta |4\rangle \quad,\quad
|l_4] = \gamma|3]
\end{equation}
Then
\begin{equation}
\begin{split}
& (l_2 - K_{23})^2 = 0 \quad\Rightarrow\quad K_{23}^2 = \alpha a [2|\slashed{K}_3|2\rangle + \alpha b [4|\slashed{K}_{23}|2\rangle
\\
& (l_2 + K_{15})^2 = 0 \quad\Rightarrow\quad K_{15}^2 = -\alpha a [2|\slashed{K}_{15}|2\rangle - \alpha b [4|\slashed{K}_{15}|2\rangle
\end{split}
\end{equation}
\begin{equation}
\begin{split}
& \alpha a = \frac{[4|\slashed{K}_{23}|4\rangle}{[24]\langle 42\rangle}
\\
& \alpha b = \frac{[32]\langle 34\rangle}{[24]\langle 42\rangle}
\end{split}
\end{equation}
On the other hand
\begin{equation}
(l_4 + K_{23})^2 = 0 \quad,\quad \beta\gamma = \frac{\langle 32\rangle}{\langle 24\rangle}
\end{equation}
Thus
\begin{equation}
c_{16}= \frac{1}{2}\frac{\langle 21\rangle^3[34][32]}{\langle 15\rangle\langle 54\rangle} = \frac{1}{2}s_{23}s_{34}A_5^{\textrm{MHV-tree}}[12345]
\end{equation}
\fi
%%%%%%%%%%%%%%%%%%%%%%%%%%
%
%
%
The last question: how about the other remaining possible intermediate species such as fermions and scalars?
These configurations should of course also be taken into account.
One might think of using the supersymmetry Ward identities~\cref{super_wi} to compute them. 
In fact, the total contribution of these configurations vanishes.
This can be done by a rather trivial analysis of the above helicity configurations.
Let us recall the action of a $\mathcal{N} = 4$ super Yang-Mills theory (cf. \eg Eq.(4.19) of~\cite{Elvang:2013cua})
\begin{equation}
S = \int \dd^4 x\tr\Big(
-\frac{1}{4}F_{\mu\nu}F^{\mu\nu} - \frac{1}{2}\big(D\Phi_I)^2 + \frac{i}{2}\bar{\Psi}\slashed{D}\Psi + \frac{g}{2}\bar{Psi}\Gamma^I \big[\Phi_I, \Psi\big] + \frac{g^2}{4}\big[\Phi_I, \Phi_J\big]^2\Big) 
\end{equation}
where $\Phi_I$'s are six real scalar field, fermions are represented by the ten-dimensional Majorana-Weyl spinor $\Psi$, $D$ is the usual covariant derivative and $F_{\mu\nu}$ represents the gluon field strength.
When considering complex scalar fields made of the real scalar ones,
we notice that, at tree-level, complex scalars and fermions always show up in pair with their conjugate fields, \ie
with opposite helicity in the out-going convention,
when coupled to gluons. 
However, all the non-vanishing helicity configurations given above involve at least one tree-level amplitude with the same helicity for the cut propagator in the product~\cref{box_coeff}.
Because it is impossible to have such a tree-level contribution with other intermediate species than gluons,
we have only to consider consider configurations with intermediate gluons.
\\\\
In consequence, the five-point MHV all-gluon amplitude $A_5^{\mathrm{1-loop}}[--+++]$ is given by
\begin{equation}\label{final_result_a5mhv}
A_5^{\mathrm{1-loop}}[--+++] = c_1 I_{1} + c_8 I_8 + c_{10}I_{10} + c_{11}I_{11} + c_{16}I_{16}
\end{equation}
where the $I_{k}$'s are the corresponding box integral for each configuration.
This result is in agreement with~\cite{Bern:1994zx}.














