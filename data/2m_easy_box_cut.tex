\section{2m easy box cut}
Consider an 2m easy box.
The massive external legs are $K_1$ and $K_3$.
The cut momenta are denoted $l_1 = l$ and $l_2 = -( l-K_1-K_2)$.
We work in the center-of-mass frame of $K_{12}$
The cut diagram is given by

\begin{equation*}
\begin{split}
\int\frac{\dd^{D-1}l_1}{(2\pi)^{D-1}} &
 \int\frac{\dd^{D-1}l_2}{(2\pi)^{D-1}}
\frac{\delta^{D}(l_1 + l_2 - K_1- K_2)}{8(l_1\cdot K_2)(2l_1\cdot K_3 + (K_3)^2)l_1^0 l_2^0} 
 = 
\int\frac{\dd^{D-1}l_1}{(2\pi)^{D-1}} 
\frac{\delta(l_1^0 + l_2^0 - K_1^0 - K_2^0)}{8 l_1^0 l_2^0 (2l_1\cdot K_3 + (K_3)^2)(l_1\cdot K_2)}
\\
& = \int\frac{|\vec{l}_1|^{D-2} \dd |\vec{l}_1|}{(2\pi)^{D-1}(2l_1^0 K_2^0)(K_{12}^0)^2}
\frac{\dd \Omega_{D-1} \delta(l_1^0 - \frac{K^0_{12}}{2})}{(2l_1^0 K_3^0 - 2l_1^0 \hat{l}\cdot\vec{K_3} + (K_3)^2)(1-\hat{l}\cdot \hat{K}_2)}
\end{split}
\end{equation*}

Let us now focus on the second piece of the above integral, which involves scalar products. 
We choose $\hat{K}_2$ to be the axis and we denote $\cos\psi = \hat{K}_2\cdot\hat{K}_3$. 
Using $\dd\Omega_{D-1} = \dd \Omega_{D-2}(1-z^2)^{\frac{D-4}{2}}\dd z$, where $z = \cos\theta$, 
\begin{equation*}
\begin{split}
& \int\frac{\dd \Omega_{D-1}}{(K_{12}^0 K_3^0 - K_{12}^0 |\vec{K}_3| \hat{l}\cdot \hat{K}_3 + (K_3)^2)(1-\hat{l}\cdot\hat{K}_2)}
\\
= & \frac{1}{(K_3)^2 + K_{12}^0 K_3^0}\int \dd \Omega_{D-2}
\int_0^\pi\dd\theta\frac{\cos\theta\sin^{-2\epsilon}\theta}{(1-\cos\theta)(1-a\cos\psi\cos\theta + a\sin\psi\sin\theta)}
\end{split}
\end{equation*}
where 
\begin{equation*}
a := \frac{|\vec{K}_3|K_{12}^0}{(K_3)^2 + K_{12}^0 K_3^0}
\end{equation*}
The integral \wrt $\theta$ can be done by a change of variable to $t = \tan\frac{\theta}{2}$, which equals to
\begin{equation*}
\begin{split}
& \int_0^\infty\dd t \frac{4t}{(1+t^2)^2}\Big(\frac{2t}{1+t^2}\Big)^{-\epsilon}\frac{1}{\Big(1-\frac{1-t^2}{1+t^2}\Big)\Big[1-\frac{a\cos\psi}{1+t^2}(1-t^2) + \frac{2a\sin\psi t}{1+t^2}\Big]}
\\
 =  &
2^{1-\epsilon}\Omega_{2-2\epsilon}
\int^{\infty}_0 \dd t t^{1-\epsilon}(1+t^2)^\epsilon
\Big[\frac{1}{(1-a\cos\psi)t^2} + \frac{B}{t} + \frac{Ct + D}{(1+a\cos\psi)t^2 + 2at\sin\psi  + 1-a\cos\psi}\Big]
\end{split}
\end{equation*}
where $B$, $C$ and $D$ are constants. 
We do not need their explicit expressions since these terms lead to contributions at the order $\mathcal{O}(\epsilon^0)$.
As a result, the cut integral reads
\begin{equation*}
\frac{K_{12}^{-2\epsilon}2^{-\epsilon}\pi^{1-\epsilon}}{\Gamma(1-\epsilon)K_{12}^0K_2^0}\frac{1}{\big[(K_3)^2 + K_{12}^0K_3^0\big](1-a\cos\psi)}
\Big(-\frac{1}{\epsilon} + \mathcal{O}(1)\Big)
\end{equation*}
In terms of the invariants
\begin{equation*}
\begin{split}
\frac{1}{\big[(K_3)^2 + K_{12}^0K_3^0\big](1-a\cos\psi)} 
= &
\frac{1}{K_3^2 + K_{12}\cdot K_3 + \frac{K_{12}^2}{K_{12}\cdot K_2}
\Big[K_3\cdot K_2 - \frac{(K_3\cdot K_{12})(K_2\cdot K_{12})}{K_{12}^2}}
\\
=&
\frac{K_{12}\cdot K_2}{(K_2\cdot K_{12})(K_3^2 + K_{12}\cdot K_3) + K_{12}^2(K_3\cdot K_2) - (K_3 \cdot K_{12})(K_2\cdot K_{12})}
\\
=&
\frac{K_1 \cdot K_{12}}{- K_1^2 K_3^2 + K_{12}^2 K_{14}^2}
\end{split}
\end{equation*}
where the following relations are used
\begin{equation*}
\begin{split}
& K_2\cdot K_3 = \frac{1}{2} (K_{14}^2  - K_3^2)  \\
& K_3\cdot K_{12} = \frac{1}{2}(K_{13}^2 - K_3^2 - K_1^2 + K_{14}^2 - K_3^2)
\\
& K_2\cdot K_{12} = \frac{1}{2}(K_{12}^2 - K_1^2)
\end{split}
\end{equation*}
As a consistency check, we can compare the result to the imaginary part of the two-mass-easy box integral given in~\cite{Britto:2010xq}.
%
%
%%%%%%%%%%%%%%%%%%%%%%%%%%%%%%%%%%%%%%%%%%%%%%%%%%%%%%%%
%%%%%%%%%%%%%%%%%%%%%%%%%%%%%%%%%%%%%%%%%%%%%%%%%%%%%%%%%%%%
%
%
%Not the best way to do... disgarded%
\color{gray}
The integration can be done in using a Feynman parametrization.
Taking into account the constraint due to the $\delta$-function, 
it becomes

\begin{equation*}
\frac{\dd \Omega_{D-1}}{(K_{12}^0 K_3^0 - K_{12}^0 |\vec{K}_3| \hat{l}\cdot \hat{K}_3 + (K_3)^2)(1-\hat{l}\cdot\hat{K}_2)}
= \frac{1}{(K_3)^2 + K_{12}^0 K_3^0}\int \dd \Omega_{D-1}\int_0^1\frac{\dd x}{\Big(1-\hat{l}\cdot\big(ax \hat{K}_3 + (1-x)\hat{K}_2\big)\Big)^2}
\end{equation*}

where 
\begin{equation*}
a := \frac{|\vec{K}_3|K_{12}^0}{(K_3)^2 + K_{12}^0 K_3^0}
\end{equation*}

Using $\dd\Omega_{D-1} = \dd \Omega_{D-2}(1-z^2)^{\frac{D-4}{2}}\dd z$, where $z = \cos\theta$ takes into account the angular dependence of the denominator, the two pieces of integral read

\begin{equation*}
\begin{split}
\int \dd \Omega_{D-1} &
\int_0^1\frac{\dd x}{\Big(1-\hat{l}\cdot\big(ax \hat{K}_3 + (1-x)\hat{K}_2\big)\Big)^2}
= \Omega_{D-2}\int_0^1\dd x \int_{-1}^1 \dd z (1-z)^{-\epsilon}(1+z)^{-\epsilon} (1-bz)^{-2}
\\
& = 
\Omega_{D-2}\int\dd x \int^1_0\dd u 2^{1-2\epsilon} u^{-\epsilon}(1-u)^{-\epsilon}(1+b)^{-2}\big(1-\frac{2bu}{1+b}\big)^{-2}
\\
& = 
\Omega_{D-2}\int^1_0 \dd x 2^{1-2\epsilon}(1+b)^{-2} (_2F_1)\big(2,1-\epsilon, 2-2\epsilon, \frac{2b}{1+b}\big)
\frac{\Gamma^2(1-\epsilon)}{\Gamma(2-2\epsilon)} 
\end{split}
\end{equation*}

where 
\begin{equation*}
b := \big| ax \hat{K_3} + (1-x)\hat{K}_2\big|
\end{equation*}
We can use the identity to get the $\mathcal{O}(\epsilon^{0})$ contribution of the hypergeometric function
\begin{equation*}
(_2F_1)(2,1-\epsilon, 2-2\epsilon, z) = (1-z)^{-1-\epsilon}(_2F_1)(-2\epsilon, 1-\epsilon, 2-2\epsilon, z)
=(1-z)^{-1-\epsilon}\big(1+\mathcal{O}(\epsilon)\big)
\end{equation*}

\color{black}


