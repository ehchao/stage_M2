%\section{2m easy box cut}
Now, let us verify the Cutkosky rule with the easy-two-mass box integral.
We apply a cut in the $s$-channel.
We work in the center-of-mass frame of $K_{12}$
The cut integral is given by
\begin{equation}\label{cuti4}
\begin{split}
\Delta I_4 =&
\int\frac{\dd^{D-1}l_1}{(2\pi)^{D-1}} 
 \int\frac{\dd^{D-1}l_2}{(2\pi)^{D-1}}
\frac{(2\pi)^D\delta^{D}(l_1 + l_2 - K_1- K_2)}{8(l_1\cdot K_2)(2l_1\cdot K_3 + (K_3)^2)l_1^0 l_2^0} 
\\
\iffalse
 = & 
\int\frac{\dd^{D-1}l_1}{(2\pi)^{D-2}} 
\frac{\delta(l_1^0 + l_2^0 - K_1^0 - K_2^0)}{8 l_1^0 l_2^0 (2l_1\cdot K_3 + (K_3)^2)(l_1\cdot K_2)}
\\
\fi
 = & \int\frac{|\vec{l}_1|^{D-2} \dd |\vec{l}_1|}{(2\pi)^{D-2}(2l_1^0 K_2^0)(K_{12}^0)^2}
\frac{\dd \Omega_{D-1} \delta(l_1^0 + l_2^0 - K^0_{12})}{(2l_1^0 K_3^0 - 2l_1^0 \hat{l}\cdot\vec{K_3} + (K_3)^2)(1-\hat{l}\cdot \hat{K}_2)}
\end{split}
\end{equation}
Let us now focus on the second piece of the above integral, which involves scalar products. 
We parametrize the spatial components $\hat{K}_2$, $\hat{K}_3$ and $\hat{l}$ in the following way
\begin{equation}\label{spatial_param}
\begin{split}
& \hat{K}_2 = (\cos\psi, 0, \sin \psi, \mathbf{0}_{D-4})
\\
& \hat{K}_3 = (1,0,0,\mathbf{0}_{D-4})
\\
& \hat{l} = (\sin\theta\cos\chi, \sin\theta\sin\chi,\cos\theta)
\end{split}
\end{equation}
Using $\dd\Omega_{D-1} = \dd \Omega_{D-2}(1-z^2)^{\frac{D-4}{2}}\dd z$, where $z = \cos\theta$, 
we have
\begin{equation}\label{omega11}
\begin{split}
& \int\frac{\dd \Omega_{D-1}}{(K_{12}^0 K_3^0 - K_{12}^0 |\vec{K}_3| \hat{l}\cdot \hat{K}_3 + (K_3)^2)(1-\hat{l}\cdot\hat{K}_2)}
\\
= & \frac{1}{(K_3)^2 + K_{12}^0 K_3^0}\int \dd \Omega_{D-3}
\int_{-1}^1 \dd(\cos\theta)\int_{-1}^1\dd(\cos \chi)\frac{\sin^{-2\epsilon}\theta \sin^{-1-2\epsilon}\chi}{(1-a\cos\theta)(1-\cos\psi\cos\theta  - \sin\psi\sin\theta\cos\chi)}
\end{split}
\end{equation}
where 
\begin{equation}
a := \frac{|\vec{K}_3|K_{12}^0}{(K_3)^2 + K_{12}^0 K_3^0}
= \frac{K_{12}^2}{\big((K_3)^2 + K_{12}\cdot K_3\big)^2}\Big( -K_3^2 + \frac{(K_2\cdot K_3)^2}{K_{12}^2}\Big)
=1
\end{equation}
The above can be verified by a simple calculation.
%%%%%%%%%%%%%%%%%%%%%%%%%%%%%%%%%
\iffalse
This can be verified by computing
\begin{equation}
\begin{split}
1-a^2 = &
\frac{1}{\big((K_3)^2 + K_{12}\cdot K_3\big)^2}
\Big( \big( (K_3)^2 + K_{12}\cdot K_3 \big)^2 + K_3K_{12} - (K_3\cdot K_{12})^2
\Big)
\\
= &
\frac{1}{(K_3\cdot K_4)^2} \Big( (K_3\cdot K_4)^2 + K_3^2K_{12}^2 - (K_3^2 + K_3\cdot K_4)^2 \Big)
\\
= &
\frac{K_3^2}{(K_3\cdot K_4)^2}\Big(-K_{34}^2 + K_{12}^2\Big)
\\
= & 0
\end{split}
\end{equation}
and $a>0$.
\fi
%%%%%%%%%%%%%%%%%%%%%%%%%%%%%%%%%%%%%%%%%%%%%
A way to evaluate~\cref{omega11} is given in~\cite{Somogyi:2011ir} (in the two-denominator massless case, eqn.(49) of the reference), where some non-trivial relationships of Mellin-Barnes representations of a hypergeometric function are used.
We will simply cite the result here.
The proof of~\cref{omega11} following~\cite{Somogyi:2011ir} will be given in Appendix~\ref{appendix-omega11}.
\\
By setting
\begin{equation}
v = \frac{1}{2}(1-\cos\psi) =\frac{st - K_1^2K_3^2}{(s-K_1^2)(s-K_3^2)}
\end{equation}
we have, after using~\cref{hypergeo_tranf}, 
\begin{equation}\label{omega11}
\begin{split}
 \int \dd \Omega_{D-3} &
\int_{-1}^1 \dd(\cos\theta)\int_{-1}^1\dd(\cos \chi)\frac{\sin^{-2\epsilon}\theta \sin^{-1-2\epsilon}\chi}{(1-a\cos\theta)(1-\cos\psi\cos\theta  - \sin\psi\sin\theta\cos\chi)}
%%%%%%%%%%%%%%%%%%%%%%
%\iffalse 
%\\
%& = 
%2^{-2\epsilon}\pi^{1-\epsilon}\frac{\Gamma^2(-\epsilon)}{\Gamma(1-\epsilon)\Gamma(-2\epsilon)}{}_2F_1(1,1,1-\epsilon, 1-v)
%\\
%& = 
%2^{-2\epsilon}\pi^{1-\epsilon}\frac{\Gamma^2(-\epsilon)}{\Gamma(1-\epsilon)\Gamma(-2\epsilon)} \frac{1}{v}{}_2F_1(1,1,1-\epsilon, 1-\frac{1}{v})
%\\
%\fi
%%%%%%%%%%%%%%%%%%%%%%%%%%%%
\\
& = 
-2^{1-2\epsilon}\pi^{1-\epsilon} \frac{\Gamma(1-\epsilon)}{\Gamma(1-2\epsilon)}\frac{1}{v}\Big(\frac{1}{\epsilon}
+\ln v + \mathcal{O}(\epsilon)\Big)
\end{split}
\end{equation}
%%%%%%%%%%%%%%%%%%%%%%%
\iffalse
In terms of the kinematic invariants
\begin{equation}
\begin{split}
v= &
\frac{1}{2}\Big( 1 - \frac{1}{K_3\cdot K_4}\frac{K_{12}^2}{K_{12}\cdot K_2}\big(K_2\cdot K_3\big) - \frac{(K_3\cdot K_{12})(K_2\cdot K_{12})}{K_{12}^2} \Big)
\\
= &
\frac{1}{2}\frac{1}{(K_3\cdot K_4)(K_1\cdot K_2)}
\Big( \frac{1}{2}K_3^2 (K_{12}- K_1^2) - K_{12}^2(K_2\cdot K_3)
\Big)
\\
=&
\frac{1}{(2K_3\cdot K_4)(2K_1\cdot K_2)}\big(K_{12}^2K_{14}^2 - K_1^2K_3^2\big)
\\
 = &
\frac{st - K_1^2K_3^2}{(s-K_1^2)(s-K_3^2)}
\end{split}
\end{equation}
\fi
%%%%%%%%%%%%%%%%%%%%%%%%%%%%%%
%
All taken into account, we have
\begin{equation}
\begin{split}
\Delta I_4 = & -\frac{1}{2}\big(\frac{K_{12}^0}{2}\big)^{2-2\epsilon}\frac{1}{(2\pi)^{2-2\epsilon}}\frac{1}{s (K_{12}\cdot K_2)}2^{1-2\epsilon}\frac{1}{(-K_3\cdot K_4)}\pi^{1-\epsilon} \frac{\Gamma(1-\epsilon)}{\Gamma(1-2\epsilon)}
\frac{1}{v}\Big(\frac{1}{\epsilon} + \ln v + \mathcal{O}(\epsilon)\Big)
\\
= & 4^{-1+\epsilon}\pi^{-1+\epsilon}\frac{\Gamma(1-\epsilon)}{\Gamma(1-2\epsilon)}s^{-\epsilon}\frac{1}{st - K_1^2K_3^2}\Big(
\frac{1}{\epsilon} + \ln\big(1-\frac{K_1^2K_3^2}{st}\big)
-\ln\big(1-\frac{K_1^2}{s}\big) - \ln\big(1-\frac{K_3^2}{s}\big) + \ln\big(\frac{t}{s}\big)\Big)
\end{split}
\end{equation}
The first factor of $\frac{1}{2}$ comes from the delta function in the last line of~\cref{cuti4}.
This expression matches the branch cut across $s < 0 $ of $I_4$ to the order of $\mathcal{O}(\epsilon^0)$ given in~\cref{i4bdk}. \\\\
%%%%%%%%%%%%%%%%%%%%%%%%%%%%%%%%%%%%%%%%%%%%%%%%%%%%%%%%
%%%%%%%%%%%%%%%%%%%%%%%%%%%%%%%%%%%%%%%%%%%%%%%%%%%%%%%%%%%%
