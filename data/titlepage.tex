
\begin{titlepage}

\newcommand{\HRule}{\rule{\linewidth}{0.5mm}} % Defines a new command for the horizontal lines, change thickness here

\center % Center everything on the page
 
%----------------------------------------------------------------------------------------
%	HEADING SECTIONS
%----------------------------------------------------------------------------------------

\textsc{\LARGE {\'E}cole Normale Sup{\'e}rieure  }\\[0.3cm] % Name of your university/college
%\textsc{\LARGE INSTITUTE OF TECHNOLOGY  }\\[0.3cm]
%\textsc{\Large JALANDHAR-144011, PUNJAB(INDIA) }\\[0.3cm]
\textsc{\Large Research internship of the M2 ICFP program}\\[0.5cm] % Major heading such as course name
 % Minor heading such as course title

%----------------------------------------------------------------------------------------
%	TITLE SECTION
%----------------------------------------------------------------------------------------

\HRule \\[0.4cm]
{ \huge \bfseries Unitarity method in one-loop QCD amplitudes}\\[0.03cm] % Title of your document
\HRule \\[1.5cm]

 
%----------------------------------------------------------------------------------------
%	AUTHOR SECTION
%----------------------------------------------------------------------------------------

\begin{minipage}{0.4\textwidth}
\begin{flushleft} \large
\emph{Author:}\\
En-Hung CHAO  \\Master 2 ICFP \\theoretical physics option \\ {\'E}cole Normale Sup{\'e}rieure (Paris, France)% Your name
~
\end{flushleft}
\end{minipage}
\begin{minipage}{0.4\textwidth}
\begin{flushright} \large
\emph{Supervised by:} \\
Dr. David Kosower \\Institut de Physique Th{\'e}orique \\Commissariat {\`a} l'{\'E}nergie Atomique (Saclay, France) % Supervisor's Name 
\end{flushright}
\end{minipage}\\[1cm]

% If you don't want a supervisor, uncomment the two lines below and remove the section above
%\Large \emph{Author:}\\
%John \textsc{Smith}\\[3cm] % Your name

%----------------------------------------------------------------------------------------
%	DATE SECTION
%----------------------------------------------------------------------------------------

{\large April 3rd - June 29th, 2018 \\Research internship at \\ Institut de Physique Th{\'e}orique}, CEA Saclay\\[1cm] % Date, change the \today to a set date if you want to be precise

%----------------------------------------------------------------------------------------
%	LOGO SECTION
%----------------------------------------------------------------------------------------

\begin{minipage}{0.4\textwidth}
\begin{flushleft} \large
\centering
\includegraphics[scale=0.4]{ens_logo}\\[1cm] % Include a department/university logo - this will require the graphicx package
  
\end{flushleft}
%\end{minipage}\\[1cm]
%\begin{minipage}{0.4\textwidth}
\begin{flushright} \large
\centering
\includegraphics[scale=0.1]{cea_logo}\\[1cm] % Include a department/university logo - this will require the graphicx package
  
\end{flushright}
\end{minipage}\\[1cm]
%----------------------------------------------------------------------------------------

\vfill % Fill the rest of the page with whitespace

\end{titlepage}

%%%%%%%%%%%%%%%%%%%%%%%%%%%%%%%%%%%%%%%%%%%%%%%%%%%%%%%%%%%%%%%%%%%%%
\iffalse
%--------------------------------------
\section*{Abstract}
This work consists of two parts.
In the first one, we investigate the vacuum polarization in 1+1 dimension in the presence of a Kondo-type potential.
We apply the renormalization by point-splitting \wrt the  Hadamard parametrix to evaluate the vacuum charge density and the vacuum stress-energy tensor.
In the second part, we study the Wentzell boundary condition for massless fermions.
After proving the well-posedness of dynamical problems and the property of causal propagation under this boundary condition, we discuss the vacuum polarization and the vacuum stress-energy tensor in using the same renormalization technique as in the first part.

\section*{Résumé}
Ce travail consiste en deux parties. 
Dans la première partie, nous nous intéressons à la polarisation du vide en dimension 1+1 en présence d'un poteniel singulier dit de "type Kondo".
Nous appliquons la renormalisation par point-splitting par rapport au parametrix d'Hadamard afin d'évaluer la densité de charge du vide et le tenseur d'énergie-impulsion du vide.
Dans la seconde partie, nous étudions la condition aux limites de Wentzell pour les fermions de masse nulle.
Après avoir prouvé que les problèmes dynamiques sont bien posés sous cette condition aux limites généralisée et la propriété de propagation causale,
nous discutons de la polarisation du vide ainsi que le tenseur d'énergie-impulsion du vide en utilisant la même technique de renormalisation introduite dans la première partie.

\newpage
%-----------------------------------------
%acknowledgement
\section*{Acknowledgements}
I am very grateful to my supervisor, Dr. Jochen Zahn, for his constructive and patient mentoring during the entire period of the internship, and also for his suggestions and reviews during the writing of this report afterwards.
The hospitality of Elementary Particle Theory Group of Insitute of Theoretical Physics of University of Leipzig (Germany), directed by Prof. Stefan Hollands, is greatly appreciated.
Besides, I would also like to thank Onirban Islam (University of Leipzig), Mojtaba Taslimitehrani (University of Leipzig) and Alexandre Efremov (École Polytechnique) for useful discussions and recommendations for literatures. 
\fi
%%%%%%%%%%%%%%%%%%%%%%%%%%%%%%%%%%%%%%%%%%%%%%%%%%%%%%%%%%%%%%%%%%%%%%%%%