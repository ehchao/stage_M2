\section{Amplitudes in N=4 super Yang-Mills by unitarity method}
Bern, Dixon, Dunbar and Kosower proposed a procedure~\cite{Bern:1994zx} based on the Cutkosky rule~\cite{doi:10.1063/1.1703676} which allows to determine compute amplitudes in a gauge field theory in a more efficient way than the traditional diagrammatic methods. 
Especially, in their paper~\cite{Bern:1994zx}, they showed how the generalized unitarity can be applied to the N=4 super Yang-Mills theory thanks to the fact that only scalar box integrals involve in amplitudes after reduction.
Let us start by review some known results.
\\\\
For one-loop amplitudes, there exists a color decomposition. In particular, when all internal particles transform as color adjoints, as is the case for N=4 SYM, the result takes the form
\begin{equation*}
\mathcal{A}_n = \sum_J n_J \sum_{c=1}^{\lfloor \frac{n}{2} + 1 \rfloor + 1}{c=1}\sum_{\sigma\in \mathcal{S}_n/\mathcal{S}_{n;c}} Gr_{n;c}(\sigma) A^{[J]}_{n;c}(\sigma)
\end{equation*}   
where $J$ is the spin and the color-structure factors are defined as
\begin{equation*}
\begin{split}
& Gr_{n;1}(1) = N_c \tr (T^{a_1}\ldots T^{a_n})
\\
& Gr_{n;c}(1) = \tr (T^{a_1}\ldots T^{a_{c-1}})\tr(T^{a_c}\ldots T^{a_n})
\end{split}
\end{equation*}
In an M=4 SYM theory, the string-based method tells us that at loop-level, the loop-momentum polynomials entering in the calculation have degree four less than a scalar in the loop,\ie $m-4$ for an $m$-point contribution.
\\\\
Now, let us sketch the proof.
Using the Cutkosky rule, it is proven that N=4 SYM MHV amplitudes contain only scalar boxes. For the proof, one does first of all a cut and use the SUSY Ward identities
\begin{equation*}
\begin{split}
A^{\mathrm{tree}}(\Lambda_1^-, g_2^+, \ldots, g_j, \ldots, \Lambda_n^+)
= \frac{\langle jn \rangle}{\langle j 1 \rangle}
A^{\mathrm{tree}}(g_1^-, g_2^+, \ldots g_j^-, \ldots , g_n^+)
\\
A^{\mathrm{tree}}(\phi_1^-, g_2^+, \ldots, g_j, \ldots, \phi_n^+)
= \frac{\langle jn \rangle^2}{\langle j 1 \rangle^2}
A^{\mathrm{tree}}(g_1^-, g_2^+, \ldots g_j^-, \ldots , g_n^+)
\end{split}
\end{equation*}
Due to supersymmetry, it suffices to compute amplitudes with only gluons. 
Then, the SUSY Ward identities determine the overall factor.
With the usage of the Schouten identity and some on-shell relations on spinors, we can reduce an N=4 amplitude to a sum of box integrals. 
\\\\
By a Passarino-Veltman reduction~\cite{PASSARINO1979151}, we can rewrite an $m$-point ($m\geq 4$) tensor integral of degree $d$ ($d\geq$ 1) as a sum of $m$- and $(m-1)$-point integrals of degree $d-1$. 
Any scalar $m$-point integral can also be reduced to a sum of scalar $(m-1)$-point integrals for $m\geq 5$. 
In this way, any one-loop $m$-point integral can be reduced to a combination of tensor box integrals of degree up to $d+4-m$.
For a renormalizable gauge theory, the maximum degree of polynomial for an $m$-point integral is $m$. Therefore, amplitudes in N=4 SYM can be reduced to scalar boxes. 
The result is unique in the sense that
\begin{equation*}
\sum c_i I_i^{\mathrm{box}} =\mathrm{polynomial}
\end{equation*}
has only trivial solution for $\{c_i\}$.
This comes from the logarithms and dilogarithms in the expression of $I_i^{\mathrm{box}}$. 
But in general, there are ration terms.


