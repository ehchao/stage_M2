\section{2m easy box integral}
The two-mass-easy box integral can be computed in using the same Feynman parametrization as in the zero-mass box integral case
At a certain point, we encounter the following integral
\begin{equation*}
I = \int\dd x\dd y \dd z y^{-1-\epsilon}(1-y)^{-1-\epsilon}s^{-2-\epsilon}
\frac{1}{-\chi + A_1 x + A_3 z + C xz}
\end{equation*}
where
\begin{equation*}
\chi = \frac{t}{s}\quad, \quad
A_1 = \chi - \frac{K_1^2}{s}\quad, \quad
A_3 = \chi - \frac{K_3^2}{s} \quad,\quad
C = -1 - \chi  + \frac{K_1^2}{s} + \frac{K_3^2}{s}
\end{equation*}
Again, the $y$-integration factorizes. Let us focus on the $x$ and $z$ dependent part.
After integrating over $x$, we are left with 
\begin{equation*}
J = \int^1_0\dd z\frac{1}{(Cz + A_1)[-\chi + A_3z]^{1+\epsilon}} - 
\int^1_0\dd z\frac{1}{(Cz + A_1)[-\chi + A_1  + (A_3 + C)z]^{1+\epsilon} }
\end{equation*}
The two integrals on the \rhs can be easily computed in applying adequate changes of variables.
Let us call the first piece on the \rhs $J_1$ and the second $J_2$.
Then
\begin{equation*}
\begin{split}
J_1 = & \int^1_0 \dd z \frac{1}{A_3^{1+\epsilon}C(z + \frac{A_1}{C})(-\frac{\chi}{A_3} + z)^{1+\epsilon}} 
\\
= &
\frac{1}{A_3^{1+\epsilon}C}\Big(
-\int^1_0 \dd u \frac{1}{z_0 + z_1}\frac{(-z_0)^{-\epsilon}}{\big(1-\frac{z_0}{z_0 + z_1}u\big) u^{1+\epsilon}} +
\int^1_0\dd v \frac{(1-z_0)^{\epsilon}}{z_0 + z_1}\frac{1}{\big(1+ \frac{1-z_0}{z_1 + z_0}v\big)v^{1+\epsilon}}\Big)
\\
& = 
\frac{1}{(z_0 + z_1 )A_3^{1+\epsilon}C}
\Big[-(-z_0)^{-\epsilon}{}_2F_1\big(1,-\epsilon, 1-\epsilon; \frac{z_0}{z_0 + z_1}\big)
+ (1-z_0)^{-\epsilon}{}_2F_1\big(1, -\epsilon, 1-\epsilon; \frac{z_0 -1}{z_0 + z_1}\big)\Big]
\end{split}
\end{equation*}
where
\begin{equation*}
z_0 = \frac{\chi}{A_3} \quad, \quad z_1 = \frac{A_1}{C}
\quad,\quad
u=\frac{-z + z_0}{z_0}\quad,\quad
v=\frac{z-z_0}{1-z_0}
\end{equation*}
$J_2$ is given by the same expression up to a factor in subsituting $z_0$ by $z_2 = \frac{\chi - A_1}{A_3 + C}$. 
%
%
%
%%%%%%%%%%%%%%%%%%%%%%
\subsection{s-channel cut}
We will test the Cutkosky rule with this explicit example. 
We apply a cut in the s-channel.
Instead of doing the phase-space integral, let us represent the delta functions in the cut integral (up to a factor) in a distributional way
\begin{equation*}
\begin{split}
\Delta I_{4}^{2m e} = &
\int\prod_{i=1}^4 \dd \alpha_i \delta(\sum_{i=1}^4\alpha_i - 1)
\frac{1}{(-\alpha_1\alpha_2 t -\alpha_3\alpha_4 s- \alpha_1\alpha_4 K_1^2 - \alpha_2\alpha_3 K_3^2 - i\alpha_3\eta - i\alpha_4\eta')^{2+\epsilon}} 
- \mathrm{c.c.}
\\
 = &
\int\dd x \dd y \dd z
\frac{y(1-y)}{[sy(1-y)(-\chi + A_1x + A_3 z + Cxy) - iy(1-x)\eta - i(1-y)(1-z)\eta']^{2+\epsilon}} -\mathrm{c.c.}
\end{split}
\end{equation*}
where the same notations as before are used and the limits $\eta\rightarrow 0$ and $\eta'\rightarrow 0$ are taken at the end.
Let us first integrate \wrt $x$.
\begin{equation*}
\begin{split}
\Delta I_4^{2me} = &
\int\dd x \dd y \dd z y^{-1-\epsilon}(1-y)s^{-2-\epsilon}
\\
&
\frac{1}{\big\{(1-y)[-\chi + (A_3+C)z + A_1] - (1-x)[(1-y)(A_1 + Cz)  +i\eta]-i\frac{1-y}{y}(1-z)\eta'\big\}^{2+\epsilon}}
\\
= &
\frac{1}{1+\epsilon}
\int\dd y \dd z y^{-1-\epsilon}s^{-2-\epsilon}(1-y)[(1-y)(A_1 + Cz ) +i\eta]^{-1}
\\
& \times
\Big(-\{(1-y)[-\chi + A_1 + (A_3 + C)z ] - i\frac{1-y}{y}(1-z)\eta'\}^{-1-\epsilon} 
\\
&
+\{(1-y)(-\chi + A_3z) - i\eta - i\frac{1-y}{y}(1-z)\eta'\}^{-1-\epsilon}\Big)
\end{split}
\end{equation*}
Call the first term on the \rhs $I_1$ and the second $I_2$. Then
\begin{equation*}
\begin{split}
I_1 = &
-\frac{s^{-2-\epsilon}}{1+\epsilon}\int\dd y \dd z (1-y)^{-\epsilon}[(1-y)(A_1 + Cz) - i\eta]^{-1}
\{y[-\chi + A_1 + A_3 + C ] 
\\
& - (1-z)[y(A_3 + C) + i\eta']\}^{-1-\epsilon}
\\
= &
\frac{s^{-2-\epsilon}}{1+\epsilon}\frac{(-1)^{-1-\epsilon}(1-y)^{-\epsilon}}{(1-y )C[(A_3+C)y + i\eta']^{1+\epsilon}}
\frac{1}{z_0 + z_1}\Big[-(-z_0)^{-\epsilon}{}_2F_1(1,-\epsilon, 1-\epsilon; \frac{z_0}{z_0 + z_1})
\\
& + (1-z_0)^{-\epsilon}{}_2F_1\big(1,-\epsilon, 1-\epsilon; \frac{z_0 -1 }{z_0+z_1}\big)\Big]
\end{split}
\end{equation*}
where
\begin{equation*}
z_0 = \frac{-y}{(A_3 + C)y +i\eta'}\quad,\quad
z_1 = \frac{-(1-y)(A_1 + C)+i\eta}{(1-y)C}
\end{equation*}
As we know already from the expansion
\begin{equation*}
{}_2F_1(1,-\epsilon,1-\epsilon,a) = 1-\epsilon\ln(1-a) - \epsilon^2\mathrm{Li}(t) + \mathcal{O}(\epsilon^{3})
\end{equation*}
let us now look at the terms which have branch cuts. 
\begin{equation*}
\frac{(-1)^{-1-\epsilon}}{(z_0+z_1)(1-y)C[A_3 + C ]y + i\eta']^{1+\epsilon}}(-z_0)^{-\epsilon}
= \frac{(-y)^{-\epsilon}}{r - i\eta-i\eta'}
\end{equation*}
where
\begin{equation*}
r = (A_1 + C)(A_3 + C)+C = \frac{K^2_1K_3^2}{s^2} -\chi 
\end{equation*}
\begin{equation*}
\frac{(-1)^{-1-\epsilon}}{(z_0+z_1)(1-y)C[A_3 + C ]y + i\eta']^{1+\epsilon}}(1-z_0)^{-\epsilon}
= \frac{(y)^{-\epsilon}(\frac{K_1^2}{s}+i\eta')^{-\epsilon}}{r - i\eta-i\eta'}
\end{equation*}









