Using a Feynman parametrization, the 2-mass easy box integral with massless internal lines in $D=4-2\epsilon$ dimension corresponds to the integral (cf. right of figure~\ref{fig-cutkosky})
\begin{equation}
\begin{split}
I_4 & = \int\frac{\dd^D l }{(2\pi)^D}\frac{1}{ l^2(l-K_1)^2(l+K_4)^2 (l-K_{12}^2)}
\\
&=
3!\times\int\frac{\dd^D l }{(2\pi)^D}
\int
\frac{\prod_{i=1}^4\dd\alpha_i \delta(\sum_{i=1}^4\alpha_i -1)}{\big[\alpha_1(l-K_1)^2 + \alpha_2(l+K_4)^2 + \alpha_3 (l-K_{12}^2) + \alpha_4\big]^4}
\\
&= \frac{i\Gamma(2+\epsilon)}{(4\pi)^{2-\epsilon}}\int\prod_{i=1}^4\dd\alpha_i \delta(\sum_{i=1}^4\alpha_i -1)\frac{1}{(-\alpha_1\alpha_2 t - \alpha_3\alpha_4 s 
-\alpha_1\alpha_4 K_1^2 - \alpha_2\alpha_3 K_{3}^2)^{2+\epsilon}}
\end{split}
\end{equation}
where
\begin{equation}
s=K_{12}^2 \quad\mathrm{and}\quad t=K_{14}^2
\end{equation}
Using the following change of variables
\begin{equation}
\alpha_1 = xy \quad
\alpha_2 = z(1-y)\quad
\alpha_3 = y(1-x)\quad
\alpha_4 = (1-y)(1-z)
\end{equation}
the above integral can be written as
\begin{equation}
I_4 = \frac{i\Gamma(2+\epsilon)}{(4\pi)^{2-\epsilon}} I
\end{equation}
where
\begin{equation}\label{box_int_I}
I = \int\dd x\dd y \dd z y^{-1-\epsilon}(1-y)^{-1-\epsilon}s^{-2-\epsilon}
\frac{1}{(-1 + A_1 x + A_3 z + C xz)^{2+\epsilon}}
\end{equation}
and
\begin{equation}
\chi = \frac{t}{s}\quad, \quad
A_1 = 1 - \frac{K_1^2}{s}\quad, \quad
A_3 = 1 - \frac{K_3^2}{s} \quad,\quad
C = -1 - \chi  + \frac{K_1^2}{s} + \frac{K_3^2}{s}
\end{equation}
The $y$ integral will give us a Beta function.
Let us focus on the $x$ and $z$ dependent part of~\cref{box_int_I}.
After integrating over $x$, we are left with 
\begin{equation}
J = \frac{1}{1+\epsilon}\Big(\int^1_0\dd z\frac{1}{(Cz + A_1)[-1 + A_3z]^{1+\epsilon}} - 
\int^1_0\dd z\frac{1}{(Cz + A_1)[-1 + A_1  + (A_3 + C)z]^{1+\epsilon} }\Big)
\end{equation}
The two integrals on the \rhs can be easily computed in applying adequate changes of variables.
Let us define $J_1$ and $J_2$ as
\begin{equation}
J = \frac{1}{1+\epsilon}\big(J_1-J_2\big)
\end{equation}
Then
\begin{equation}
\begin{split}
J_1 = & \int^1_0 \dd z \frac{1}{A_3^{1+\epsilon}C(z + \frac{A_1}{C})(-\frac{1}{A_3} + z)^{1+\epsilon}} 
\\
= &
\frac{1}{A_3^{1+\epsilon}C}\Big(
-\int^1_0 \dd u \frac{1}{z_0 + z_1}\frac{(-z_0)^{-\epsilon}}{\big(1-\frac{z_0}{z_0 + z_1}u\big) u^{1+\epsilon}} +
\int^1_0\dd v \frac{(1-z_0)^{\epsilon}}{z_0 + z_1}\frac{1}{\big(1+ \frac{1-z_0}{z_1 + z_0}v\big)v^{1+\epsilon}}\Big)
\\
& = 
-\frac{1}{(z_0 + z_1 )A_3^{1+\epsilon}C\epsilon}
\Big[-(-z_0)^{-\epsilon}{}_2F_1\big(1,-\epsilon, 1-\epsilon; \frac{z_0}{z_0 + z_1}\big)
+ (1-z_0)^{-\epsilon}{}_2F_1\big(1, -\epsilon, 1-\epsilon; \frac{z_0 -1}{z_0 + z_1}\big)\Big]
\end{split}
\end{equation}
where
\begin{equation}
z_0 = \frac{1}{A_3} \quad, \quad z_1 = \frac{A_1}{C}
\quad,\quad
u=\frac{-z + z_0}{z_0}\quad,\quad
v=\frac{z-z_0}{1-z_0}
\end{equation}
$J_2$ is given by the same expression up to a factor in subsituting $z_0$ by $z_2 = \frac{1 - A_1}{A_3 + C}$. 
The hypergeometric function can be expanded as
\begin{equation}
{}_2F_1 (1,-\epsilon, 1-\epsilon, a) = 
1-\epsilon\ln(1-a) - \epsilon^2 \dilog (a) + \mathcal{O}(\epsilon^3)
\end{equation}
%%%
%%%%%%%%%%%%%%%%%%%%%%%%%%%%%%%%%%%%%
\iffalse
With this expansion, we have
\begin{equation}
\begin{split}
I  = &
-s^{-2-\epsilon}\frac{\Gamma^2(-\epsilon)}{\Gamma(-2\epsilon)}
\frac{1}{C+A_1A_3}\frac{1}{\epsilon(1+\epsilon)}\Big[
-(-1)^{-\epsilon} + \big(\frac{-K_1^2}{s}\big)^{-\epsilon} + \big(\frac{-K_3^2}{s}\big)^{-\epsilon} - (-\chi)^{-\epsilon}
\\
&+
\epsilon\Big(\ln\big(\frac{A_1A_3}{C+A_1A_3}\big) - \ln\big(\frac{A_3(C-A_1 )}{C+A_1A_3}\big) - \ln\big(\frac{A_1(C-A_3)}{C+A_1A_3}\big) +
\ln\big(\frac{(1-\chi)C + A_1A_3}{C+A_1A_3}\big) 
\Big)
\\
& +\epsilon^2\Big(
\dilog\big(\frac{C}{C+A_1A_3}\big) - \dilog\big(\frac{C(1-A_3)}{C+A_1A_3}\big) - \dilog\big(\frac{C(1-A_1)}{C+A_1A_3}\big)+\dilog\big(\frac{C(1-A_1-A_3-C)}{C+A_1A_3}\big)
\\
& +\ln(-1)\ln\big(\frac{A_1A_3}{C+A_1A_3}\big)
-\ln\big(-\frac{K_3^2}{s}\big)\ln\big(\frac{A_3(C-A_1)}{C+A_1A_3}\big)
\\
&
-\ln\big(-\frac{K_1^2}{s}\big)\ln\big(\frac{A_1(C-A_3)}{C+A_1A_3}\big)
+\ln(-\chi)\ln\big(\frac{(1-\chi)C + A_1A_3}{C+A_1A_3}\big)
\Big)
\Big]
+\mathcal{O}(\epsilon)
\end{split}
\end{equation}
\fi
%%%%%%%%%%%%%%5
%%%%%%%%%%%%%%%%%
Using the dilogarithm identity
\begin{equation}
\dilog(1-z) +\dilog(z) = -\ln(z)\ln(1-z) + \frac{\pi^2}{6}
\end{equation}
we get, in terms of invariants
\begin{equation}\label{i4final}
\begin{split}
I_4 = &\frac{2i\Gamma(1+\epsilon)}{(4\pi)^{2-\epsilon}}\frac{ \Gamma^2(1-\epsilon)}{\Gamma(1-2\epsilon)}\frac{1}{st-K_1^2K_3^2}\frac{1}{\epsilon^2}
\Big[
(-s)^{-\epsilon} - (-K_1^2)^{-\epsilon} - (-K_3^2)^{-\epsilon} + (-t)^{-\epsilon}
\\&
+\epsilon^2\Big(
\dilog(1-f) + \dilog(1-\frac{s}{t}f) - \dilog(1-\frac{K_1^2}{s}f) - \dilog(1-\frac{K_3^2}{s}f)
\Big)
\Big]
+\mathcal{O}(\epsilon)
\end{split}
\end{equation}
where
\begin{equation}
f=\frac{C}{C+A_1A_3} = s\Big( \frac{-s-t + K_1^2 + K_3^2}{-st + K_1^2K_3^2}\Big)
\end{equation}
\cref{i4final} is in agreement with~\cite{Duplancic:2000sk}. 
Another more used expression is the one given in~\cite{Bern:1993kr}
\begin{equation}\label{i4bdk}
\begin{split}
I_4 = &\frac{2i\Gamma(1+\epsilon)}{(4\pi)^{2-\epsilon}}\frac{ \Gamma^2(1-\epsilon)}{\Gamma(1-2\epsilon)}\frac{1}{st-K_1^2K_3^2}\frac{1}{\epsilon^2}
\Big[
(-s)^{-\epsilon} - (-K_1^2)^{-\epsilon} - (-K_3^2)^{-\epsilon} + (-t)^{-\epsilon}
\\&
+\epsilon^2\Big(
\dilog\big(1-\frac{K_1^2K_3^2}{st}\big) 
-\dilog\big(1-\frac{K_1^2}{s}\big) 
-\dilog\big(1-\frac{K_1^2}{t}\big) 
-\dilog\big(1-\frac{K_3^2}{s}\big) 
-\dilog\big(1-\frac{K_3^2}{t}\big) 
-\frac{1}{2}\ln^2\big(\frac{s}{t}\big)
\Big)
\Big]
\\ &
+\mathcal{O}(\epsilon)
\end{split}
\end{equation}
To get this expression, some other dilogarithm identities are needed.
This is discussed in detail in~\cite{Duplancic:2000sk}.
%
%
%
\iffalse %not used
%%%%%%%%%%%%%%%%%%%%%%
\subsection{s-channel cut}
We will test the Cutkosky rule with this explicit example. 
We apply a cut in the s-channel.
Instead of doing the phase-space integral, let us represent the delta functions in the cut integral (up to a factor) in a distributional way
\begin{equation}
\begin{split}
\Delta I_{4}^{2m e} = &
\int\prod_{i=1}^4 \dd \alpha_i \delta(\sum_{i=1}^4\alpha_i - 1)
\frac{1}{(-\alpha_1\alpha_2 t -\alpha_3\alpha_4 s- \alpha_1\alpha_4 K_1^2 - \alpha_2\alpha_3 K_3^2 - i\alpha_3\eta - i\alpha_4\eta')^{2+\epsilon}} 
- \mathrm{c.c.}
\\
 = &
\int\dd x \dd y \dd z
\frac{y(1-y)}{[sy(1-y)(-1 + A_1x + A_3 z + Cxy) - iy(1-x)\eta - i(1-y)(1-z)\eta']^{2+\epsilon}} -\mathrm{c.c.}
\end{split}
\end{equation}
where the same notations as before are used and the limits $\eta\rightarrow 0$ and $\eta'\rightarrow 0$ are taken at the end.
Let us first integrate \wrt $x$.
\begin{equation}
\begin{split}
\Delta I_4^{2me} = &
\int\dd x \dd y \dd z y^{-1-\epsilon}(1-y)s^{-2-\epsilon}
\\
&
\frac{1}{\big\{(1-y)[-1 + (A_3+C)z + A_1] - (1-x)[(1-y)(A_1 + Cz)  +i\eta]-i\frac{1-y}{y}(1-z)\eta'\big\}^{2+\epsilon}}
\\
= &
\frac{1}{1+\epsilon}
\int\dd y \dd z y^{-1-\epsilon}s^{-2-\epsilon}(1-y)[(1-y)(A_1 + Cz ) +i\eta]^{-1}
\\
& \times
\Big(-\{(1-y)[-1 + A_1 + (A_3 + C)z ] - i\frac{1-y}{y}(1-z)\eta'\}^{-1-\epsilon} 
\\
&
+\{(1-y)(-1 + A_3z) - i\eta - i\frac{1-y}{y}(1-z)\eta'\}^{-1-\epsilon}\Big)
\end{split}
\end{equation}
Call the first term on the \rhs $I_1$ and the second $I_2$. Then
\begin{equation}
\begin{split}
I_1 = &
-\frac{s^{-2-\epsilon}}{1+\epsilon}\int\dd y \dd z (1-y)^{-\epsilon}[(1-y)(A_1 + Cz) - i\eta]^{-1}
\{y[-1 + A_1 + A_3 + C ] 
\\
& - (1-z)[y(A_3 + C) + i\eta']\}^{-1-\epsilon}
\\
= &
\frac{s^{-2-\epsilon}}{1+\epsilon}\frac{(-1)^{-1-\epsilon}(1-y)^{-\epsilon}}{(1-y )C[(A_3+C)y + i\eta']^{1+\epsilon}}
\frac{1}{z_0 + z_1}\Big[-(-z_0)^{-\epsilon}{}_2F_1(1,-\epsilon, 1-\epsilon; \frac{z_0}{z_0 + z_1})
\\
& + (1-z_0)^{-\epsilon}{}_2F_1\big(1,-\epsilon, 1-\epsilon; \frac{z_0 -1 }{z_0+z_1}\big)\Big]
\end{split}
\end{equation}
where
\begin{equation}
z_0 = \frac{-\chi y}{(A_3 + C)y +i\eta'}\quad,\quad
z_1 = \frac{-(1-y)(A_1 + C)+i\eta}{(1-y)C}
\end{equation}
As we know already from the expansion
\begin{equation}
{}_2F_1(1,-\epsilon,1-\epsilon,a) = 1-\epsilon\ln(1-a) - \epsilon^2\mathrm{Li}(a) + \mathcal{O}(\epsilon^{3})
\end{equation}
let us now look at the terms which have branch cuts. 
For its usefulness, we compute
\begin{equation}
z_0 + z_1 = \frac{-(A_1 + C)(A_3 + C) - C +i\eta + i\eta'}{C(A_3 + C + i\eta')y}
\end{equation}
%
\begin{equation}
\frac{(-1)^{-1-\epsilon}}{(z_0+z_1)C[(A_3 + C )y + i\eta']^{1+\epsilon}}(-z_0)^{-\epsilon}
= -\frac{(-\chi y)^{-\epsilon}}{r - i\eta-i\eta'}
\end{equation}
where
\begin{equation}
r = -(A_1 + C)(A_3 + C)-C = -\frac{K^2_1K_3^2}{s^2} +\chi 
\end{equation}
%
\begin{equation}
\frac{(-1)^{-1-\epsilon}}{(z_0+z_1)C[(A_3 + C)y + i\eta']^{1+\epsilon}}(1-z_0)^{-\epsilon}
= - \frac{(y)^{-\epsilon}(\frac{-K_1^2}{s}+i\eta')^{-\epsilon}}{r - i\eta-i\eta'}
\end{equation}
%
\begin{equation}
\frac{z_0}{z_0 + z_1} = \frac{-C\chi }{r + i\eta + i\eta'}
\end{equation}
%
\begin{equation}
\frac{z_0 - 1}{z_0 + z_1} = \frac{C(-\frac{K_1^2}{s}+i\eta')}{r + i\eta + i\eta'}
\end{equation}
%
On the other hand, 
$I_2$ takes exactly the same form as $I_1$ up to a factor and the substition of $z_0$ by
\begin{equation}
z_2 = \frac{1-A_3 + i\eta}{A_3-i\eta'}
\end{equation}
The relevant computations are
\begin{equation}
z_1 + z_2 = \frac{-r + i\eta + i\eta'}{C(A_3 - i\eta')}
\end{equation}
%
\begin{equation}
\frac{(-1)^{-1-\epsilon}}{(z_2+z_1)C[A_3 y - i\eta']^{1+\epsilon}}(-z_2)^{-\epsilon}
= -\frac{(-\frac{K_3^2}{s} + i\eta)^{-\epsilon}}{r - i\eta-i\eta'}
\end{equation}
%
\begin{equation}
\frac{(-1)^{-1-\epsilon}}{(z_2+z_1)C[A_3y - i\eta']^{1+\epsilon}}(1-z_2)^{-\epsilon}
= - \frac{(-1-i\eta-i\eta')^{-\epsilon}}{r - i\eta-i\eta'}
\end{equation}
%
\begin{equation}
\frac{z_2}{z_2+z_1} = -\frac{C(1-A_3 + i\eta)}{r-i\eta-i\eta'}
\end{equation}
%
\begin{equation}
\frac{z_2-1}{z_2+z_1} = -\frac{C(1+i\eta+i\eta')}{r-i\eta-i\eta'}
\end{equation}
\fi


