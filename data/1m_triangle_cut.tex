%\section{1m triangle cut}
Let us first consider the one-mass triangle integral with massive external leg $K^2 \neq 0$ (left of~\ref{fig-cutkosky})
\begin{equation*}
I_3 = \int\frac{\dd^D l }{(2\pi)^D}\frac{1}{l^2(l-K)^2(l+K_2)^2}
\end{equation*}
As mentioned earlier, we use the dimensional regularization scheme with $D = 4-2\epsilon$.
This integral can be computed using the standard techniques: Feynman parametrization and Wick rotation (since the integrant vanishes at infinity)
\begin{equation*}
\begin{split}
I_3 = & \int\frac{\dd^D l }{(2\pi)^D}\int_0^{1} \dd x \int_0^{1-x}\dd y \frac{2!}{(l^2 - x^2 K^2 + xK^2 + 2xy K\cdot K_2)^3}
\\
= &
-\frac{i}{(4\pi)^{2-\epsilon}}\Gamma (1+\epsilon)
\int^1_0 \dd x\int_0^1 \dd u x^{-1-\epsilon} (1-x)^{-1-\epsilon}
\frac{1}{(-K^2 - 2uK\cdot K_2)^{1+\epsilon}}
\\
= &
\frac{i}{(4\pi)^{2-\epsilon}}\frac{\Gamma^2(1-\epsilon)\Gamma(1+\epsilon)}{\Gamma(1-2\epsilon)}
\frac{1}{\epsilon^2}(-K^2)^{-1-\epsilon}
\end{split}
\end{equation*}
where a change of variable $u =\frac{1}{(1-x)}y$ is used in the second line and $K_3^2 = 0$ is used to obtain the final result.
\\
Now, let us do a cut on the $K_1^2$-channel. 
The uncut internal leg carries momentum $l + K_2$. 
We choose to work in the center-of-mass frame of the massive leg.
%
\begin{equation*}
\begin{split}
\Delta I_3^{1m} = \int\dd \Pi_{\textrm{LIPS}}(l, l-K) \frac{1}{(l+K_2)^2} & =
\int\frac{\dd^{D-1}l_1}{(2\pi)^{D-1}}\int\frac{\dd^{D-1}l_2}{(2\pi)^{D-1}}
\frac{(2\pi)^{D}\delta^{D}(l_1 + l_2 - K)}{8(l\cdot K_2)l_1^0 l_2^0}
\\
& = \int\frac{\dd^{D-1}l_1}{(2\pi)^{D-2}}\frac{\delta(l_1^0 + l_2^0 - K^0)}{2(l\cdot K_2)(K^0)^2} 
\\
& = \int\frac{|\vec{l}|^{D-2}\dd |\vec{l}|}{(2\pi)^{D-2}} \frac{\dd\Omega_{D-1}\delta(l_1^0 + l_2^0 - K^0)}{2 ( l \cdot K_2)(K^0)^2}
\\
& = \frac{1}{2}\frac{\Omega_{D-2}\big(\frac{K^0}{2}\big)^{D-2}}{(2\pi)^{D-2}(K_0)^3 K^0_2} \int_{-1}^1\dd z (1-z)^{-1-\epsilon}(1+z)^{-\epsilon}
\end{split} 
\end{equation*}
The factor of $\frac{1}{2}$ in the last line comes from the delta function in the second last line.
%
The last integral gives a beta-function
%
\begin{equation*}
\begin{split}
\int^1_{-1} (1-z)^{-1-\epsilon}(1+z)^{-\epsilon} \dd z 
=\int^1_0(2s)^{-\epsilon} 2^{-1-\epsilon} (1-s)^{-1-\epsilon} 2 \dd s
=2^{-2\epsilon}\frac{\Gamma(-\epsilon)\Gamma(1-\epsilon)}{\Gamma(1-2\epsilon)}
\end{split}
\end{equation*}
%
As a result, the cut in terms of the invariants is
\begin{equation*}
\Delta I_3^{1m} = 
\frac{2\pi^{\frac{D}{2}-1}}{\Gamma(\frac{D}{2}-1)}\frac{\big(\frac{K^2}{4}\big)^{\frac{D-2}{2}}}{(2\pi)^{D-2}K^2(K_2\cdot K)} 2^{-2\epsilon}\frac{\Gamma(-\epsilon)\Gamma(1-\epsilon)}{\Gamma(1-2\epsilon)}
=
-\frac{1}{(4\pi)^{1 - \epsilon}\epsilon}(K^{2})^{-1-\epsilon}\frac{\Gamma(1-\epsilon)}{\Gamma(1-2\epsilon)}
\end{equation*}
We verify that, up to the order of $\mathcal{O}(\epsilon^{-1})$, this corresponds to the branch cut of $I_3$ across the branch $K^2 > 0$.