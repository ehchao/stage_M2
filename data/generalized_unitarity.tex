\section{Generalized unitarity}
Bern, Dixon, Dunbar and Kosower proposed a procedure~\cite{Bern:1994zx} based on the Cutkosky rules~\cite{doi:10.1063/1.1703676} which allows to determine compute amplitudes in a gauge field theory in a more efficient way than the traditional diagrammatic methods. 
Especially, in their paper~\cite{Bern:1994zx}, they showed how the generalized unitarity can be applied to the N=4 super Yang-Mills theory thanks to the fact that only scalar box integrals involve in amplitudes after reduction.
\\\\
Based on Landau's discussion on the singularities of the amplitudes calculated from an arbitrary Feynman diagram~\cite{LANDAU1959181}, 
Cutkosky proposed a generalized version of the unitarity condition~\cite{doi:10.1063/1.1703676} which is based on the discontinuity across a cut starting from any of Landau's branch points.
\\
The simplest physical cut is a double cut (or a two-particle cut).
Let us consider a generic one-loop amplitude with explicit dependence on the kinematic invariant $K^2$
\begin{equation*}
A(K^2) = \int\frac{\dd^D l}{(2\pi)^D} A_L\frac{1}{l^{2} (l-K)^{2}}A_R
\end{equation*}
where $A_{L,R}$ are rational fractions in the loop momentum $l$ and the external momenta. 
In Cutkosky's language, a double cut amounts to choose two of the propagators in the loop on-shell.
Mathematically, this means doing the following propagator subsitutions\footnote{One should be careful with the $i\epsilon$ prescription which is often omitted in amplitude computations}
\begin{equation}\label{on-shell_propagator}
\frac{1}{l^2 + i\epsilon} \rightarrow 2\pi i\delta^{(+)}\big(l^2\big)
\quad,\quad
\frac{1}{(l-K)^2 + i\epsilon} \rightarrow 2\pi i\delta^{(+)}\big((l-K)^2\big)
\end{equation}
We call this a double cut in the $K^2$-channel.
Cutkosky's rules states that we will get the branch cut, \ie the discontinuity of $A$ between approching the branch $K^2$ from above and approching from below on the complex plan
\begin{equation}\label{cutkosky}
A(K^2 + i\epsilon) - A(K^2 - i\epsilon) =
-4\pi^2 \int\frac{\dd^D l}{(2\pi)^D}A^{\mathrm{tree}}_LA^{\mathrm{tree}}_R \delta^{(+)}(l^2)\delta^{(+)}\big((l-K)^2\big) 
\end{equation}
Here, $A_L$ and $A_R$ become genuine amplitudes at tree-level because of the on-shell internal propagators. 
\\\\
We illustrate~\cref{cutkosky} by two different examples.
%
%
\subsection{Example: cut one-mass triangle}
%\section{1m triangle cut}
Let us first consider the one-mass triangle integral with massive external leg $K^2 \neq 0$ (left of~\ref{fig-cutkosky})
\begin{equation*}
I_3 = \int\frac{\dd^D l }{(2\pi)^D}\frac{1}{l^2(l-K)^2(l+K_2)^2}
\end{equation*}
As mentioned earlier, we use the dimensional regularization scheme with $D = 4-2\epsilon$.
This integral can be computed using the standard techniques: Feynman parametrization and Wick rotation (since the integrant vanishes at infinity)
\begin{equation*}
\begin{split}
I_3 = & \int\frac{\dd^D l }{(2\pi)^D}\int_0^{1} \dd x \int_0^{1-x}\dd y \frac{2!}{(l^2 - x^2 K^2 + xK^2 + 2xy K\cdot K_2)^3}
\\
= &
-\frac{i}{(4\pi)^{2-\epsilon}}\Gamma (1+\epsilon)
\int^1_0 \dd x\int_0^1 \dd u x^{-1-\epsilon} (1-x)^{-1-\epsilon}
\frac{1}{(-K^2 - 2uK\cdot K_2)^{1+\epsilon}}
\\
= &
\frac{i}{(4\pi)^{2-\epsilon}}\frac{\Gamma^2(1-\epsilon)\Gamma(1+\epsilon)}{\Gamma(1-2\epsilon)}
\frac{1}{\epsilon^2}(-K^2)^{-1-\epsilon}
\end{split}
\end{equation*}
where a change of variable $u =\frac{1}{(1-x)}y$ is used in the second line and $K_3^2 = 0$ is used to obtain the final result.
\\
Now, let us do a cut on the $K_1^2$-channel. 
The uncut internal leg carries momentum $l + K_2$. 
We choose to work in the center-of-mass frame of the massive leg.
%
\begin{equation*}
\begin{split}
\Delta I_3^{1m} = \int\dd \Pi_{\textrm{LIPS}}(l, l-K) \frac{1}{(l+K_2)^2} & =
\int\frac{\dd^{D-1}l_1}{(2\pi)^{D-1}}\int\frac{\dd^{D-1}l_2}{(2\pi)^{D-1}}
\frac{(2\pi)^{D}\delta^{D}(l_1 + l_2 - K)}{8(l\cdot K_2)l_1^0 l_2^0}
\\
& = \int\frac{\dd^{D-1}l_1}{(2\pi)^{D-2}}\frac{\delta(l_1^0 + l_2^0 - K^0)}{2(l\cdot K_2)(K^0)^2} 
\\
& = \int\frac{|\vec{l}|^{D-2}\dd |\vec{l}|}{(2\pi)^{D-2}} \frac{\dd\Omega_{D-1}\delta(l_1^0 + l_2^0 - K^0)}{2 ( l \cdot K_2)(K^0)^2}
\\
& = \frac{1}{2}\frac{\Omega_{D-2}\big(\frac{K^0}{2}\big)^{D-2}}{(2\pi)^{D-2}(K_0)^3 K^0_2} \int_{-1}^1\dd z (1-z)^{-1-\epsilon}(1+z)^{-\epsilon}
\end{split} 
\end{equation*}
The factor of $\frac{1}{2}$ in the last line comes from the delta function in the second last line.
%
The last integral gives a beta-function
%
\begin{equation*}
\begin{split}
\int^1_{-1} (1-z)^{-1-\epsilon}(1+z)^{-\epsilon} \dd z 
=\int^1_0(2s)^{-\epsilon} 2^{-1-\epsilon} (1-s)^{-1-\epsilon} 2 \dd s
=2^{-2\epsilon}\frac{\Gamma(-\epsilon)\Gamma(1-\epsilon)}{\Gamma(1-2\epsilon)}
\end{split}
\end{equation*}
%
As a result, the cut in terms of the invariants is
\begin{equation*}
\Delta I_3^{1m} = 
\frac{2\pi^{\frac{D}{2}-1}}{\Gamma(\frac{D}{2}-1)}\frac{\big(\frac{K^2}{4}\big)^{\frac{D-2}{2}}}{(2\pi)^{D-2}K^2(K_2\cdot K)} 2^{-2\epsilon}\frac{\Gamma(-\epsilon)\Gamma(1-\epsilon)}{\Gamma(1-2\epsilon)}
=
-\frac{1}{(4\pi)^{1 - \epsilon}\epsilon}(K^{2})^{-1-\epsilon}\frac{\Gamma(1-\epsilon)}{\Gamma(1-2\epsilon)}
\end{equation*}
We verify that, up to the order of $\mathcal{O}(\epsilon^{-1})$, this corresponds to the branch cut of $I_3$ across the branch $K^2 > 0$.
\subsection{Example: cut two-mass-easy box}
%
In the previous examples, we see that the branch cuts come from the logrithms and dilogrithms in the scalar triangle and box integral. 
In effect, the basis integrals in~\cref{master_equation} do have branch cuts which can be distinguished from the different kinematic invariant dependences between them.
\color{red}point to a ref\color{black}
In effect, one can use double cuts to reconstruct an amplitude if the rational term in~\cref{master_equation} is missing since it has no branch cuts.
This class of amplitudes is called \textit{cut-constructible}~\cite{Bern:1994cg}. 
\\\\
A more systematic way to determine the coefficients in~\cref{master_equation} is to use quadruple~\cite{BRITTO2005499} and triple cuts~\cite{Forde:2007mi}, which is a generalization of the Cutkosky rule.