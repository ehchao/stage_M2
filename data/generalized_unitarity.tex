\section{Generalized unitarity}
Bern, Dixon, Dunbar and Kosower proposed a procedure~\cite{Bern:1994zx} based on the Cutkosky rules~\cite{doi:10.1063/1.1703676} which allows to determine compute amplitudes in a gauge field theory in a more efficient way than the traditional diagrammatic methods. 
Especially, in their paper~\cite{Bern:1994zx}, they showed how the generalized unitarity can be applied to the N=4 super Yang-Mills theory thanks to the fact that only scalar box integrals involve in amplitudes after reduction.
\\\\
Based on Landau's discussion on the singularities of the amplitudes calculated from an arbitrary Feynman diagram~\cite{LANDAU1959181}, 
Cutkosky proposed a generalized version of the unitarity condition~\cite{doi:10.1063/1.1703676} which is based on the discontinuity across a cut starting from any of Landau's branch points.
