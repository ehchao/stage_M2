\section{Summary}
In this report, we have reviewed some modern scattering amplitude computational methods. 
The usage of spinor-helicity allows to have compact expressions.
In gauge theories, we may decompose, in certain cases, an amplitude into a color-dependent part and a color-ordered partial amplitude which depends only on kinematics.
Color-ordered amplitudes can be determined using the BCFW recursion, which exploits the holomorphicity of amplitudes with shifted momenta. 
As for one-loop, the existence of a basis for loop integrals, the master integras, has been known since long time ago and the reduction techniques have also been well understood. 
Methods based on generalized unitarity, inspired by the Cutkosky rules, provide efficient way of extracting coefficients of an amplitude in the basis comprised of master integrals. 
In four dimensions, the maximal number of cuts is four and box coefficients can be easily determined. 
Triangle and bubble coefficients can be determined by triple and double cuts with well-chosen loop momentum parametrizations.
In a dimensionally regularized theory, there may still be some ambiguities due to the rational term. 
In the very end of this report, a rational term extraction algorithm, which makes use of unitarity cuts and similar loop-momentum parametrizations as in the triple cut case, has been briefly presented. 
A continuation of this internship will be applying generalized unitarity to two loops, which is the current frontier of the research in this area.   
\\\\
The graphs of this report are generated using Jaxodraw~\cite{Binosi:2003yf}, based on Axodraw~\cite{1994CoPhC..83...45V}.

